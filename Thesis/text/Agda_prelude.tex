Although functional languages have been receiving great attention
and a fast evolution in recent years, one of the biggest jumps
has been the introduction of dependent types. Agda\cite{norell07} is one such language, other 
big contributions being \emph{Epigram} \cite{mcbride05} and \emph{Coq} \cite{bertot06}. 

In languages such as Haskell or ML, where a Hindley-Milner based algorithm is used
for type checking, values and types are clearly separated. Within dependent
types, though, the story changes. A type can depend on any value the programmer
wishes. Classical examples are the fixed size vectors \inlagda{Vec\; A\; n}, where \inlagda{A}
is the type of the elements and \inlagda{n} is the length of the vector. Readers familiar
with C may argue that C also has fixed size arrays. The difference is that
this \inlagda{n} need not to be known at compile time: it can be an arbitrary term , as long as it has type \inlagda{Nat}. That is, there is no difference between types and values, they are all sets, explained in the sequel.

This chapter introduces the basics of Agda and shows some examples in both
the programming and the proof theoretic sides of the language. Later on, in 
section \ref{sec:background:martinlof}, we will see the theory in which Agda is built 
on top of, whereby a number of concepts introduced below will become clearer. 
Some background on the $\lambda$-calculus and the Curry-Howard isomorphism is
presented in section \ref{sec:background:lambdacalculus}.

\subsection{Peano Naturals}

In terms of programming, Agda and Haskell are very close. Agda's syntax was
inspired by Haskell and, being also a functional language, a lot of
the programming techniques we need to use in Haskell also apply for an Agda program.
Knowledge of Haskell is not strictly necessary, but of a great value for a better
understating of Agda.\\

The \emph{Hello World} program in Agda is the encoding of Peano's Natural numbers, which follows:

\Agda{Basic}{NAT}

Data declarations in Agda are very similar to a GADTs\cite{Xi2003} in Haskell. Remember that, in Agda, types and values are the same thing.
The data declaration above states that \inlagda{Nat} is \emph{of type} 
\inlagda{Set}. This \inlagda{Set} is the type of types, to be addressed in more detail later. 
For now, think of it as Haskell's $*$ kind.


As expected, definitions are made by structural induction over the
data type. Alas in Haskell, pattern matching is the mechanism of choice here.

\Agda{Basic}{PLUS-MUL}

There are a few points of interest in the above definitions. The underscore pattern
(\_) has the same meaning as in Haskell, it matches anything. The underscore in
the symbol name, though, indicates where the parameters should be relative to the symbol name. 
Agda supports mixfix operators and has full UTF8 support. It is possible to
apply a mixfix operator in normal infix form, for instance: \inlagda{a + b = \_+\_\; a\; b}.

\subsection{Propositional Logic, a fragment}

Let us continue our exposition of Agda by encoding some propositional logic. 
First of all two sets are needed for representing the truth values:

\Agda{Basic}{TOP-BOT}

The truth proposition has only one proof, and this proof is trivial. So we define $\top$ as a set
with a single, constant, constructor. The absurd is modeled as a set with no constructors, therefore no elements. In the theory of types,
this means that there is no proof for the proposition $\bot$, as expected. Note the definition
of \N{$\bot\hyph elim$}, whose type is $\bot \rightarrow A$, for all sets $A$\footnote{
%%%%% BEGIN FOOTNOTE
Implicit parameters such as \inlagda{ \{ A : Set \} } are enclosed in brackets. This can be read as universal quantification in
the type level.
%%%%% END FOOTNOTE
}. This exactly captures
the notion of proof by contradiction in logic. The definition is trickier, though.
Agda's empty pattern, \inlagda{()}, is used to discharge a contradiction. It tells
that there is no possible pattern in such equation, and we are discharged of writing
a right-hand side.

%The top set, $\top$, on the other
%hand, is written as a set with only one constructor, proving $\top$ is trivial. Yet, for reasons that will be clear 
%later, we chose to encode $\top$ as a record with no fields, and therefore only one constructor, \inlagda{record\{\}} in Agda's syntax.
%Following Agda's standard library notation, let's call it \D{Unit}.

%\Agda{Basic}{UNIT}

Let us now show how to encode a fragment of propositional logic in this framework. Taking conjunction as a first candidate, we begin by declaring the set that represents conjunctions:

\Agda{Basic}{CONJUNCTION}

So, \D{A /\symbol{92} B} contains elements with a proof of $A$ and a proof of $B$. Its elements
can only be constructed with the \IC{<\_,\_>} constructor. The reader familiar with Natural Deduction
might recognize the following trivial properties. Given $D$ a set:

\begin{enumerate} %[i)]
  \item \emph{Formation} rules for $D$ express the conditions under which $D$ is a set.
        In our example, whenever $A$ and $B$ are sets, then so is \D{A /\symbol{92} B}.
  \item \emph{Introduction} rules for $D$ define the set $D$, that is, they specify how the
        canonical elements of $D$ are constructed. For the conjunction case, 
        the elements of \D{A /\symbol{92} B} have the form \IC{< a , b >}, where $a \in A$
        and $b \in B$. 
  \item \emph{Elimination} rules show how to prove a proposition about an arbitrary element in $D$.
        They are very closely related to structural induction. The Agda equivalent is pattern matching,
        where we deconstruct, or eliminate, an element until we find the first constructor. Note that
        the proofs of \F{/\symbol{92}-elim} are simply pattern matching.
  \item \emph{Equality} rules give us the equalities which are associated with $D$. Without diving
        too much, in the simple case above, \inlagda{< a , b > == < a' , b' >} whenever \inlagda{a == a'}
        and \inlagda{b == b'}.
\end{enumerate}

As detailed below in section \ref{sec:background:martinlof}, these properties come for free whenever we introduce 
a new set forming operation, that is, a data declaration in Agda. In fact, each set forming operation $D$ offers the four kinds or rules just given.

An analogous technique can be employed to encode disjunction. Note the similarity with
Haskell's \D{Either} datatype (they are the same, in fact). The elimination rule for disjunction is a proof by cases. 

\Agda{Basic}{DISJUNCTION}

As a trivial example, we can prove that conjunction distributes
over disjunction.

\Agda{Basic}{DISTR}

Using this fragment of natural deduction, we can start proving things. Consider a simple
proposition saying that an element of a nonempty list $l_1 \conc l_2$, where $\conc$ is concatenation, either is
in $l_1$ or in $l_2$. For this all that is required is to: encode lists in Agda, provide a definition for $\conc$,
encode a \emph{view} of the elements in a list and finally prove our proposition.

Defining the type of lists is a boring task, and the definition is just like
its Haskell counterpart. One can just import the definition from the standard library and go to the
interesting part right away.

\Agda{Basic}{IN-LIST}

Looking closely to this code, we one finds a set forming operation \D{In}, that receives
one implicit parameter $A$ and is indexed by an element of $A$ and a $List\ A$. Indeed, \DArgs{In}{x\; l}
is the proposition that states that $x$ is an element of $l$. How is such a proof constructed?
There are two ways of doing so. Either $y$ is in $l$'s head, as captured by the \IC{InHead}
constructor, or it is in $l$'s tail, and a proof of such statement is required, in the \IC{InTail}
constructor. The reader may wonder about the statement \DArgs{In}{y\; []}, which should be impossible to construct. One can see,
just from the types, that we cannot construct such a statement. The result of both \IC{In} constructors
involve non-empty lists and they are the only constructors we are allowed to use. So, the \emph{non-empty list}
requirement in the proposition we are trying to prove comes for free.

We proceed to showing how to state the proposition mentioned above and carry out its proof step by step, as this is an interesting example.
The proposition is stated as follows:

\Agda{Basic}{inDistr-decl}

The variables enclosed in normal parentheses are called \emph{telescopes}, and they introduce
the dependent function type $(x : A) \rightarrow B\;x$, in general, $x$ can appear on the right-hand side of
the function type constructor, $\rightarrow$. In this proof, we need an implicit set $A$, two lists of $A$,
an element of $A$ and a proof that such element is in the concatenation of the two lists.
The proof follows by \emph{induction} on the first list. 

\Agda{Basic}{inDistr-1}

If the first list is empty, $[] \conc l_2$ reduces to $l_2$. So, prf has type \DArgs{In}{x\; l_2}, which
is just what we need to create a disjunction. If not empty, though, then it has a head and a tail:

\Agda{Basic}{inDistr-2}

 Now we have to also pattern-match
on $prf$. If it says that our element is in the head of our list, the result is also very simple.
The repetition of the symbol $x$ in the left hand side is allowed as long as all duplicate occurrences
are preceded by a dot. These are called dotted patterns and tell Agda that this is \emph{the only possible}
value for that pattern. We can see this by reasoning with \IC{InHead} type, \DArgs{In}{x\; (x :: l)}. 
So, if $l_1 = (x :: l)$ and we have $prf = In\; x\; (x :: l)$, that is, a proof that $x$ is in $l_1$'s head,
we can only be looking for $x$. 

\Agda{Basic}{inDistr-3}

In case our element is not in $l_1$'s head, it might be either in the rest of $l_1$ or in $l_2$.
Note that we pass a \emph{structurally smaller} proof to the recursive call. This is a condition required by
Agda's termination checker.

\subsection{Finite Types}

The theory of types is not only roses. The soundness requirement is that every function must
terminate. In fact, Agda passes all definitions through its termination checker and if
it does not recognize a definition as terminating, it will output an error.

An interesting technique to write terminating functions is to use finite types. One example
where we need to use such finite types is first-order unification, as in \cite{mcbride03}.
As the unification algorithm progresses, the terms are growing bigger, since variables are being
substituted by (possibly) more complex terms. Yet, the number of variables to be substituted
decreases by one in each step. This is modeled by a function with type $\D{Fin}\;(\IC{suc}\;n) \rightarrow \D{Fin}\;n$,
where \D{Fin $n$} is a type with $n$ elements. 

Agda, and dependently typed languages in general, allows one to speak about type \emph{families}
in a very structured way. To define the family $(\D{Fin}\;n)_{n \in \mathbb{N}}$ of types with 
\emph{exactly} $n$ elements we proceed as follows.

\Agda{Fin}{fin-def}

If we wish to enumerate, for instance, the inhabitants of type \D{Fin 3}, we would find
that they are: \IC{fz}, \IC{fs fz} and \IC{fs fs fz}. Now, how many inhabitants does \D{Fin 0} have?
It is indeed an uninhabited type, which is congruent with the intuition of having \emph{exactly} zero elements.

The careful reader might have noticed a slight syntax difference between \D{Fin}'s definition
and the other datatypes presented before. In general, Agda allows one to define datatypes
with parameters $p_1, \cdots, p_n$ and indices $i_1 , \cdots, i_k$. The parameters works just like
a Haskell type parameter. Indices, on the other hand, is what allows us to define inductive families
in Agda. As stated in the excerpt above, the \emph{type} of \D{Fin} is \D{$\mathbb{N} \rightarrow$ Set}.
In a \K{data} declaration, parameters are the variables before the colon, whereas indices appear after the colon.
Vectors of a fixed length are another famous inductive family example. 

\subsection{Closing Remarks}

The small introduction to Agda given above tells a tiny bit of what the language is capable of,
both in its programming side and its proof assistant side. All the concepts were introduced
informally here, with the intention of not overwhelming a unfamiliar reader. A lot of resources
are available in the Internet at the Agda Wiki page \cite{AgdaTutorials}. 



