What we call $\lambda$-calculus is a collection of various formal systems based 
in the notation invented by Alonzo Church in \cite{Church01,Church02}. 
Church solved the famous \textit{Entscheidungsproblem} (from German, \textit{decision problem}) 
proposed by David Hilbert, in 1928. The challenge consisted in providing an algorithm capable
of determining whether or not a given mathematical fact was valid in a given language. 
Church proved that there is no solution for such problem, that is, it's an undecidable problem.

One of Church's main objectives was to build a formal system for the foundations of
Mathematics, just like Martin-Löf's type theory, which was be presented later, around 1970. 
Church dropped his work when his basis was found to be inconsistent. Later on it was found
that there were ways of making it consistent again, with the help of types.

The notion of type is paramount for this thesis as a whole. This notion arises when we want 
to combine different terms in a given language, for instance, it makes no sense to try
to compute $\int \mathbb{N}\;\mathrm{d}x$. Although syntatically correct, it's subterms have
different \emph{types} and, therefore, are not compatible. A type can be seen as a
categorization of terms.

For a torough introduction of the lambda-calculus, the reader is directed to \cite{Barendregt01,Hindley01}.
The goal of this chapter is to make it easier to the reader to see how Martin-Löf's type theory
was born. There are a lot of connections with it's cousin the $\lambda$-calculus and seeing this
connections might give a better understanting. 

\subsection{The $\lambda$-calculus}

\begin{mydef}[Lambda-terms] Let $\mathcal{V} = \{v_1, v_2, \cdots\}$ be a infinite set of
variables, $\mathcal{C}=\{c_1, c_2, \cdots\}$ a set of constants such that 
$\mathcal{V} \cap \mathcal{C} = \emptyset$. The set $\Lambda$ of lambda-terms is 
inductively defined by:
\begin{description}
  \item[Atoms]
        \[ \mathcal{V} \cup \mathcal{C} \subset \Lambda \]
  \item[Application]  
        \[ \forall M, N \in \Lambda.\; (M N) \in \Lambda \]
  \item[Abstraction]
        \[ \forall M \in \Lambda, x \in \mathcal{V}.\; (\lambda x . M) \in \Lambda \]
\end{description}
\end{mydef}

Let's make some conventions that will be usefull throughout this document. We'll usually denote
terms by uppercase letters $M, N, O, P, \cdots$ and variables by lowercases $x, y, z, \cdots$.
Application is left associative, that is, the term $M N O$ represents $((M N) O)$ whereas
abstractions are right associative, so, $\lambda x y. K$ represents $(\lambda x . (\lambda y.K))$.

Suppose we have a term $\lambda yz.x(yz)$, we say that variables $y$ and $z$ are bounded variables
and $x$ is a free variable (there is no abtraction binding it visible). For now on, we'll 
use Barendregt's convenssion for variables and assume that terms do not have any name clashing.
In fact, whenever we have two terms $M$ and $N$ that only differ in the naming of their variables,
for instance $\lambda x . x$ and $\lambda y . y$, we say that they are $\alpha$-convertible.\\

\begin{mydef}[Substitution] Let $M$ and $N$ be lambda-terms where $x$ has a free occurence
in $M$. The substitution of $x$ by $N$ in $M$, denoted by $[N/x]M$ is, informally, the result
of replacing every $x$ in $M$ by $N$. It is defined by induction on $M$ by:
\begin{eqnarray*}
   {[N/x]} x & = & N \\
   {[N/x]} y & = & y, \; y \neq x \\
   {[N/x]}(M_1 M_2) & = & {[N/x]M_1\; [N/x]M_2} \\
   {[N/x]}(\lambda y . M) & = & (\lambda y . {[N/x]M})
\end{eqnarray*}
\end{mydef}

\subsection{Beta reduction and Confluency}

We are now equiped with both a notion of term and a formal definition of substitution, we can
model the notion of computation. The intuitive meaning is very simple. Imagine a normal
function $f(x) = x + 3$ and suppose we want to compute $f(2)$. All we have to do is to
substitute $x$ for $2$ in the body of $f(x)$, resulting in $2+3$.

This notion is followed to the letter in the $\lambda$-calculus. A term with the form $(\lambda x . M)N$ is
called a $\beta$-redex and can be reduced to $[N/x]M$. If a given term has no $\beta$-redexes we say
it is in $\beta$-normal form.

\newcommand{\betaright}{\rightarrow_{\beta}}
\newcommand{\betarightright}{\twoheadrightarrow_{\beta}}
\begin{mydef}[$\beta$-reduction]
Let $M, M'$ and $N$ be lambda-terms and $x$ a variable. Let $\betaright$ be the following binary relation
over $\Lambda$ defined by induction on $M$.
\begin{center}
    \begin{tabular}{c c}
      \infer{(\lambda x . M) N \betaright [N/x]M}{} & 
      \infer{(\lambda x.M) \betaright (\lambda x. M')}{M \betaright M'} \\[1cm]
      \infer{MN \betaright M'N}{M \betaright M'} & 
      \infer{NM \betaright NM'}{M \betaright M'} \\[1cm]
    \end{tabular}
  \end{center}
\end{mydef}

We denote by $M \betarightright N$ when $N$ is obtained through zero or more $\beta$-reductions 
from $M$.\\

\begin{mydef}[$\beta$-equality]
Let $M$ and $N$ be lambda terms, we say that $M$ and $N$ are $\beta$-equal, and denote by 
$M =_{\beta} N$ if $M \betarightright N$ or $N \betarightright M$.\\
\end{mydef}

\begin{thm}[Confluency]
Let $M, N_1$ and $N_2$ be lambda terms. If $M \betaright N_1$ and $M \betaright N_2$ then there
exists a term $Z$ such that $N_i \betarightright Z$, for $i \in \{1, 2\}$.\\
\end{thm}

\begin{thm}[Church-Rosser]. Let $M$ and $N$ be lambda-terms such that $M =_{\beta} N$. Then there
exists a term $Z$ such that $M \betarightright Z$ and $N \betarightright Z$.
\end{thm}

Note that the aforementioned results are of enormous relevance not only for the $\lambda$-calculus,
but for similar formalisms too. They allow us to prove that, for instance, the normal form
of a lambda-term (if it exists\footnote{%
%%%%% BEGIN FOOTNOTE
There are terms that do not have a normal form. A classical example is $(\lambda x . xx)(\lambda x . xx)$.
The reader is invited to compute a few $\beta$-reductions on it.
%%%%% END FOOTNOTE
}) is unique. In fact, lambda-calculus consistency is proved using these results \cite{Barendregt01}.

\subsection{Simply typed $\lambda$-calculus}

\begin{mydef}[Type]
Let $\mathcal{C}_{\mathcal{T}} = \{ \sigma, \sigma', \cdots \}$ be a set of atomic types,
we define the set $\mathbb{T}$ of simple types by indution:
\begin{enumerate}[i)]
  \item $\mathcal{C}_{\mathcal{T}} \subset \mathbb{T}$
  \item $\forall \sigma, \tau \in \mathbb{T}. \; (\sigma \rightarrow \tau) \in \mathbb{T}$.
\end{enumerate}
\end{mydef}

In a programming context, we're surrounded by variable declarations. Some languages (the strongly-typed ones)
expect some information about the type of such variables. This is what we call a context.
Formally, a context is a set $\Gamma \subseteq \mathcal{V} \times \mathbb{T}$, whose elements
are denoted by $(x : \sigma)$.

This allows us to define the notion of derivation and derivability, and we're almost closing the
gap between programming and logic.\\

\begin{mydef}[Derivation]
\label{def:simpletypederivation}
We define the set of all type derivations by induction in the target lambda-term:
\begin{enumerate}[i)]
  \item
    \[\vcenter{\infer[(Ax)]{\Gamma \vdash x:\sigma}{}} \]
    
  \item 
    \[\vcenter{\infer[(I\rightarrow)]{\Gamma \vdash (\lambda x.M) : (\tau \rightarrow \sigma)}
							{\infer*{\Gamma, x:\tau \vdash M:\sigma}{}}} \]

  \item 
    \[\vcenter{
			\infer[(E\rightarrow)]
				{\Gamma \vdash MN : \sigma}
				{
					\infer*
						{\Gamma \vdash M:(\tau \rightarrow \sigma)}
						{}
				&
					\infer*
						{\Gamma \vdash N:\tau}
						{}
				}
		
		}\]
\end{enumerate}
\end{mydef}

\begin{mydef}[Derivability]
Let $\Gamma$ be a context, $M$ a lambda-term and $\sigma$ a type. We say that
the sequent $\Gamma \vdash M : \sigma$ is derivable is there exsits a derivation with
such sequent as it's conclusion.
\end{mydef}

The simply-typed $\lambda$-calculus is a model of computation. It has the same expressive power as
the Turing Machine for expressing computability notions. This is a very well studied subject
and the references provided in this chapter are a compilation of everything that
has been studied so far. For more typed variations of the $\lambda$-calculus we refer the reader
to \cite{Barendregt03}. We're not interested in that aspect of the $\lambda$-calculus, 
though. We want to explore it's connection with Mathematical logic. In the next subsection
we'll explore this connection.

\subsection{The Curry-Howard Isomorphism}

On one hand we have the models of computation, on the other hand we have the proof systems.
A a first glance, they look like very different formalisms, but they turned out to be
structurally the same. Let $M$ be a term and $\Gamma$ a context such that
$\Gamma \vdash M : \sigma$ is derivable. We can look at $\sigma$ as a propositional formula\footnote{
%%%% BEGIN FOOTNOTE
Remeber that the implication, here denoted by $\subset$, forms a minimal complete connective set
and is, therefore, enough to express the whole propositional logic.
%%%% END FOOTNOTE
} and to $M$ as a proof of such formula. There are other ways to show this connection,
but I'll illustrate it using the Natural Deduction\cite{Prawitz01} (we'll denote the
propositional implication by $\supset$). Let's put the rules presented in definition
\ref{def:simpletypederivation} side-by-side with the Axiom, $\supset$-elimination and
$\supset$-introduction rules from Natural Deduction;

\begin{center}
\begin{align*}
	\text{Natural Deduction} & \hspace{3cm} \text{Type Derivation} \\
	\sigma \hspace{1.4cm} & \hspace{3cm} \infer[(Ax)]{\Gamma \vdash M : \sigma}{}  \\[0.5cm]
	\infer[(I\supset)]{\tau \supset \sigma}{\infer*{\sigma}{[\tau]}}
		& \hspace{3cm} 
		\infer[(I\rightarrow)]{\Gamma \vdash (\lambda x . M) : (\tau \rightarrow \sigma)}
							  {\Gamma, x:\tau \vdash M : \sigma} \\[0.5cm]					  
	\infer[(E\supset)]{\sigma}{\tau \supset \sigma & \tau}
		& \hspace{3cm}
		\infer[(E\rightarrow)]{\Gamma \vdash MN : \sigma}
			{ \Gamma \vdash M : (\tau \rightarrow \sigma) & \Gamma \vdash N : \sigma}
\end{align*} 
\end{center}

This semingly shallow equivalence is a remarkable result in Computing Science, discovered
by Curry and Howard in \cite{Curry01,Howard01}. This was the starting point for the
first proof checkers, since checking a proof is the same as typing a lambda-term. If the
term is typeable, then the proof is valid. Up to this point we only presented the simpler
version of this connection. We'll later on build on top of it and add all ingredients
for working over first-order logic and using Agda as our proof assistant. Under the curtains,
all Agda does is type-checking terms.

The understanding of this connection is of big importance for writing proofs and programs in
Agda (or any other proof-assistant based on the Curry-Howard isomorphism, for that mater).


