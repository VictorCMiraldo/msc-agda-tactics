With the beginning of formal logic being traced back to Aristotle, most of the
groundbreaking work was done around the end of the 19th and early 20th centirues.
We saw the specification of propositional calculus; what we now know as predicate logic; between others. The quest for bootstrapping Mathematics, that is, formalize Mathematics in Formal Logic, was
being porsued by many. A notorious atempt was made by Russel and Whitehead in \emph{Principia Mathematica} (1910), where they believed that all mathematical truths
could be derived from inference rules and axioms, therefore opening up the question of automated reasoning. Yet, in 1931, Kurt G\"{o}del published his famous first and second incompletness theorems. In a (very small) nutshell, they state that there are some truths that are not provable, regardless of the axiomatic system chosen. This question was further addressed by Alonzo Church and Alan Turing, in the late 1930s. That's when we saw a definite notion of computability arising (in fact they gave two, independent, definitions). 

Given a formula in a logic system, the question of whether or not such formula is true can vary from trivial to impossible. The simplest case is, of course, propositional logic, where validity is decidable but not at all that interesting
for software verification in general. We need formal systems that are more expressive in order to encode software specifications, as they usually involve quantification or even modal aspects. 

\begin{TODO}
  \item some glue here...
  \item tools evolved given the presentation of more expressive systems?
\end{TODO}

Instead of completely automated reasoning, which is very hard (if not impossible) to achieve, we could only provide a guiding hand to our fellow mathematicians. That's infact what we call a \emph{Proof Assistant}.

\begin{TODO}
  \item finish...
\end{TODO}

\begin{TODO}
  \item Assisted program construction became the same thing as assisted theorem proving.
  \item And that's what we're doing here! :-)
\end{TODO}
