As we could see in section \ref{sec:prelude:agdalanguage}, Agda is a very expressive language
and it allows us to build smaller proofs than the great majority of proof assistants available.
The mixfix feature gives the language a very customizable feel, one application is the
equational reasoning framework. In the following illustration we prove the associativity of
the concatenation operation.

\Agda{Basic}{++-assocH}

The notation is very clear and understandable, it indeed looks very much
to what a \emph{squiggolist}\footnote{
%%%% BEGIN FOOTNOTE
\emph{Squiggol} is a slang name for the Bird–Meertens formalism, due to the squiggly symbols it
uses.
%%%% END FOOTNOTE
} would write on paper. One of the main downsides to it, which is also inherent to Agda in general,
is the need to specify every single detail of the demonstration, even the trivial ones. Note the
trivial, yet explicit, $(\IC{\_∷\_} x)$ congruence beeing stated.

Aiming somewhat higher, we could actually generalize the congruences to substitutions, as long as the
underlying equality exibits a substitutive behavior. We can borrow a excerpt from a relational
proof that the relation twice, defined by $ (a, 2\times a) \in twice $ for all $a \in \mathbb{N}$,
preserves even numbers. (the actual code is much larger and will be ommited, this is for illustration
purposes only)

\Agda{CaseStudy1}{twice-even-proof}

Besides the obvious Agda boilerplate, it's simple to see how the substitutive\footnote{
%%% BEGIN FOOTNOTE
As we shall see later, this is not the case, and this example was heavily modified due to
the encoding of relational equality not beeing a substitutive relation in Agda.
%%% END FOOTNOTE
} behavior of Relational
Equality can become a burden to write in every single step, but such factor is also what
allows us to rewrite arbitrary terms in a formula. The main idea is to provide a automatic
mechanism to infer the first argument for the \F{substitute} function, making equational reasoning much
cleaner in Agda.

Alongisde the development of such functionality, based on Agda's reflection capabilities (that is,
to access and modify a program AST in compile time), we also want to develop a library for
Relational Algebra, focused on it's equational reasoning aspect. Both tasks have to walk
with hands tied, since their design is mutually dependent as I shall illustrate later.

The work documented by this thesis is, therefore, split into two main tasks:
\begin{enumerate}[i)]
  \item Provide a library for Relational Algebra that is suitable for
  \item Equational Reasoning with automatic substitution inference.
\end{enumerate}

It is worth mentioning that although Relations are our main case study, we want
our substitution inference to work independently of the proof context.





