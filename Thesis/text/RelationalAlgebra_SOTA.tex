The current developments within Relational Algebra in Agda vary a lot, mainly due to
different goals. One has a plethora of options when developing a library. It is therefore
important to weight the options with the goals. As far as we know, at the time
of writing this thesis, there was no library that fitted our specific goals, which
are: (A) An expressive encoding of relational algebra; and (B) easy to handle in the meta-level,
for automatic rewritting.

The library constructed by Kahl, \cite{RATHAgda}, is not specific to Relational Algebra, but
instead he chose to construct a library for Category Theory, from which one can use
the specific category of relations, \catname{Rel}. The theories in RATH-Agda are intended
for a high-level programming, not so much focused on theorem proving. We can see that
the equality problem we ran into is not handled in Kahl's library, it is not even needed there.
\begin{TODO}
  \item How about using Kahl's library as our backend? And, later on, providing just
        a wrapper to prevent evaluation and use Kahl's?
\end{TODO}

Another interesting library is the one developed by Jansson, Mu and Ko at \cite{Jansson09}. 
Their goal, however, still is very different from both RATH-Agda and this project.
Mu, et al., are focussing on deriving programs given specifications, which is a big merit of
functional programming. Indeed, they explore the equational reasoning of programs in order to keep
\emph{refining} programs. They also avoid the non-extensional equality problem by defining
another equality type and proving its substitutiveness whenever necessary. Their main focus
is on subrelation-reasoning, though.

\begin{TODO}
  \item Expand this... maybe a few excerpts from both libraries?
\end{TODO}
