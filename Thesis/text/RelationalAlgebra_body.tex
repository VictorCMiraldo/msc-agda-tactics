Unlike most programming languages, an encoding of Relational Algebra in a dependently typed 
language allows one to truly see the advantage of dependent types in action. Most of the definitions
happen at the \emph{type level} (remember that there is no difference between types and values in Agda,
this is just a mnemonic to help understanding the \emph{functions}). The encoding presented
below is based on \cite{Jansson09}, the most significant differences being due to
extensional equality.

Long story short, a binary relation $R$ of type $A \rightarrow B$ can be thought of in terms of several mathematical objects.
The usual definition is to say that $R \subseteq A \times B$, where $\cdot\times\cdot$ is the cartesian product of sets.
In fact, $R$ contains pairs or related elements whose first component is of type $A$ and second component is of type $B$.
Note this is more general than the concept of a function, where we can have $b_1\;R\;a$ and $b_2\;R\;a$, which means
that $(a, b_1) \in R$ and $(a, b_2) \in R$, as a perfectly valid relation, but not a function, since $a$ would be mapped to two different values, $b_1$ and $b_2$.

\newcommand{\powerset}{\mathcal{P}}
Another way of speaking about relations, though, is to consider functions of type $A \rightarrow \powerset B$.
If our previous $R$ is to encoded as a function $f$, we will then have $f\; a = \{b_1, b_2\}$. For the more
mathematically inclined reader, the arrows $A \rightarrow \powerset B$ in the category \catname{Sets}, of sets and
functions, correspond to arrows $A \rightarrow B$ in the category \catname{Rel}, of sets and relations. For this matter,
we actually call \catname{Rel} the Kleisli Category for the monad $\powerset$. 
In fact, we can make the encoding of relations as functions more explicit
by defining the so-called powerset transpose of a relation $R$, which is the function that for each $a$
returns the exact (possibly empty) subset of $B$ that is related to $a$ trough $R$. This encoding is central
in the Algebra of Programming book \cite{Bird97}.

\[ (\Lambda\; R)\; a = \{ b \in B \;\mid\; b\;R\;a \} \]

For our Agda encoding of Relations, we shall use a slight modification of the "$\powerset$ approach".
We begin by encoding set theoretic notions, the most important of all being undoubtedly
the membership notion $\cdot \in \cdot$. One way of encoding a subset of a set $A$ is using a function $f$
of type $A \rightarrow Bool$, the subset beeing obtained by $\{\; a \in A\; |\; f\;a\; \}$. Yet, in Agda, this would
force that we deal only with decidable domains, which is not a problem if everything is finite, but
for infinite domains this would not work.

Another option, which is the one used in \cite{Jansson09}, is to use a function $f$
of type $A \rightarrow \D{Set}$ to encode a subset of $A$. Remember that \D{Set} is
the type of types in Agda. Although not very intuitive, this is much more expressive than the 
last option, whereby the induced subset is defined by $\{\; a \in A\; |\; f\;a\;\text{is inhabited }\}$,
which turns out to be:

\Agda{RelationsCore}{subset-def}

Extending from sets to binary relations is a very simple task. Besides the canonical steps,
we'll also swap the arguments for a relation, following what is done in \cite{Jansson09} and
keeping the syntax closer to what one would write on paper, since we usually write a relational
statement from left to right, that is, $y\;R\;x$ means $(x, y) \in R$.
\begin{eqnarray*}
  \powerset (A \times B) & = & \powerset (B \times A)\\
                         & = & B \times A \rightarrow \D{Set}\\
                         & = & B \rightarrow A \rightarrow \D{Set}
\end{eqnarray*}

Below we present the base encoding of relations in Agda following this plan, together with a few constructs
to help in their definition and to illustrate usage.

\Agda{RelationsCore}{relation-def}

The operations are defined just as we would expect them to be. Note that \D{\_⊎\_} represents a disjoint union of sets
(equivalent to Either in Haskell) and \D{\_×\_} represents the usual cartesian product. The function
lifting might look like the less intuitive construction, but it's a very simple one. $\mathit{fun}\;f$ is a relation,
therefore it takes a $b$ and a $a$ and should return a type that is inhabited if and only if $(a, b) \in fun\;f$,
or, to put it another way, if $b \equiv f\;a$, almost like its textbook definition.

\section{Relational Equality}

A notion of convertibility is paramount to any form of rewriting. In out scenario, whenever
two relations are equal, we can substitute them back and forth in a given equation. There are
a couple different, but equivalent notions of equality. The simplest one is borrowed from set
theory.\\

\begin{mydef}[Relation Inclusion]
Let $R$ and $S$ be suitably typed binary relations, we say that $R \subseteq S$ 
if and only if for all $a , b$, if $b\;R\;a$ then $b\;S\;a$.\\
\end{mydef}

\begin{mydef}[Relational Equality]
For $R$ and $S$ as above, we say that $R \releq S$ if and only if
$R \subseteq S$ and $S \subseteq R$.\\
\end{mydef}

\begin{lemma}
Relation Inclusion is a partial order, that is, $\subseteq$ is reflexive, transitive and anti-symmetric.
And $\releq$ is an equivalence relation (reflexive, transitive and symmetric).\\
\end{lemma}

So far so good. At first sight one would believe that there are no problems with such definitions
(and in fact there will be no problem as long as we keep away from Agda, but then it would not be fun, right?). 
Definitions were wrapped as datatypes in Agda to prevent expansion when using reflection,
this implies that we have some boilerplate code that must be coped with.

\Agda{RelationsCore}{subrelation}

The reader is invited to take a deeper look into what this definition means. 
Let $R : \D{Rel}\;A\;B$, for $A$ and $B$ \D{Set}s, $a : A$ and $b : B$. The statement
$R\;b\;a$ yields a \D{Set}, and an element $r : R\;b\;a$ represents
a proof that $b$ is related to $a$ trough $R$. We have no interest in the contents of
such proof though, as its existence is what matters. In more formal terms, we would like
that $\D{Rel}$ returned a proof irrelevant set. In Coq this would be \texttt{\small Prop}, but
Agda has no set of propositions.

Such a problem comes to the surface once we try to use Agda's propositional equality 
instead of $\releq$ (and hence the name distinction):

\Agda{RelationsCore}{releq-is-not-propeq}

We have to use functional extensionality anyway (here, disguised as \F{rel-ext}). But the 
type of our goal is $R\;b\;a \equiv S\;b\;a$ and the proof gets stuck since set equality is
undecidable. What we want is that $R$ and $S$ be inhabited at
the same time.

So, it is clear that we also need some encoding of $\releq$:

\Agda{RelationsCore}{releq}

But this definition is not substitutive in Agda. Therefore we need to lift it to
a substitutive definition. In fact, it is safe to lift it to the standard propositional equality.
The advantages of doing so are that we win all of the functionality to work with $\equiv$ for free.

The solution we are currently adopting relies on using yet another disguised version of
function extensionality (here as powerset transpose extensionality) 
and on postulating the proof-irrelevant notion of set equality (as in Set Theory):

\Agda{RelationsCore}{releq-lifting}

The powerset transpose postulate is pretty much function-extensionality with some syntactic sugar, and it is
known to not introduce any contradiction. But $ℙ\hyph ext$ on the other hand, has to be justified.
The reason for such a postulate is just because somewhere in our library, we need a notion
of proof-irrelevance able to state relational equality. As discussed in the next section,
there are a couple places were we can insert this notion, but postulating it at the subsets ($ℙ$) level is
by far the easier to handle and introduces the least amount of boilerplate code. (Remember that
if $a$ is a subset of $A$, that is $a : ℙ A$, stating $a\;x$ is stating $x \in a$, for any $x : A$.).

There are a few other options of achieving a substitutive behavior for $\releq$,  
as we shall present in the next sections.

\subsection{Hardcoded Proof Irrelevance}

  This idea of a proof irrelevant set was mentioned before, as was the fact that we postulated
  such notion, which is not the ideal thing to do in Agda. We want to keep our postulates to an
  absolute minimum.
  
  In order to formally talk about proof irrelevance we need some concepts from HoTT, the Homotopy Type Theory\cite{hottbook}.
  HoTT is an immense and complex newly founded field of Mathematics, and we are not going to explain
  it in detail. The general idea is to use abstract Homotopy Theory to interpret types. In normal
  type theory, a statement $a : A$ is interpreted as $a$ is an element of the set $A$. In HoTT, however,
  we say that $a$ is a point in space $A$. Objects are seen as points in a space and the type $a = b$ is seen as a path 
  from point $a$ to point $b$.
  It iss obvious that for every point $a$ there is a path from $a$ to $a$, therefore $a = a$, giving us reflexivity.
  If we have a path from $a$ to $b$, we also have one from $b$ to $a$, resulting in symmetry.
  \emph{Chaining} of paths together corresponds to transitivity. So we have an equivalence relation $=$.
  
  Even more important, is the fact that if we have a proof of $P\; a$ and $a = b$, for a predicate $P$, we can
  transport this proof along the path $a = b$ and arrive at a proof of $P b$. This corresponds to
  the substitutive behavior of standard equality. Note though, that in classical mathematics,
  once an equality $x = y$ has been proved, we can just switch $x$ and $y$ wherever we want. 
  In HoTT, however, there might be more than one path from $x$ to $y$, and they might yield different
  results. So, it is important to state which path is being \emph{walked through} when we use substitutivity.
  
  The first important notion is that of homotopy. Traditionally, we take that two functions are
  the same if they agree on all inputs. A homotopy between two functions is very close to that
  classical notion:
  
  \Agda{ProofIrrel}{homotopy-def}
  
  It is worth mentioning that \F{\_\~{}\_} is an equivalence relation, although we omit the proof.
  We follow by defining an equivalence, which can be seen an encoding of the notion of isomorphism:
  
  \Agda{ProofIrrel}{isequiv-def}
  
  That is, $f$ is an equivalence if there exists a $g$ such that $g$ is a left and right-inverse
  of $f$.
  
  Now we can state when two types are equivalent. For the reader familiar with algebra, it is
  not very far from the usual isomorphism-based equivalence notion. As expected, univalence is
  also an equivalence relation.
  
  \Agda{ProofIrrel}{univalence-def}
  
  Following the common practice when encoding HoTT in Agda, we have to postulate the Univalence
  axiom, which in short says that univalence and equivalence coincide:
  
  \Agda{ProofIrrel}{univalence-axiom}
  
  Now, our job becomes much easier, and it suffices to show that if two relations are
  mutually included, then they are univalent.
  
  \subsubsection{Mere Propositions}
  
  As we mentioned earlier, proof irrelevance is a desired property in most systems. In HoTT,
  one distinguishes between mere propositions and other types, where mere propositions
  are defined by:
  
  \Agda{ProofIrrel}{isprop-def}
  
  This allows us to state some very useful properties, which leads us to handle propositions
  as both true or false, depending on whether or not they're inhabited. These corresponds to
  lemma 3.3.2 in \cite{hottbook}.
  
  \Agda{ProofIrrel}{lemma-332}\\
  \Agda{ProofIrrel}{not-lemma-332}
  
  Both are provable in Agda, but the proofs are omitted here.
  
  \subsubsection{Adding Relations to the mix}
  
  To exploit such fine treatment of equality in our favor, we need to add
  relation and a few other details to the mix. The relation definition
  given before is kept, while pushing to the users the responsibility of proving that their relations
  are both mere propositions and decidable.
  
  This can be easily done with Agda's instance mechanism:
  
  \Agda{ProofIrrel}{instances}
  
  This will treat both records as typeclasses in the Haskell sense. Now, for talking about
  subrelations they must be an instance of \D{IsProp}, and whenever we wish to use anti-symmetry
  they must also be an instance of \D{IsDec}. It turns out that anti-symmetry is now
  provable as long as we assume relational extensionality.
  
  \Agda{ProofIrrel}{subrel-def}\\
  \Agda{ProofIrrel}{subrel-antisym}
  
  Although we could arrive at the result we desired with a minimal number of postulates (univalence axiom, only), 
  the user would be heavily punished when wanting to define a relation, not to
  say that decidability would give problems once relational composition (discussed below) enters the stage. For this reasons
  we chose not to adopt this solution \emph{as is}, even though it is more formal than what we previously
  presented.

\subsubsection{Custom Universes}


  We could remove the \D{IsProp} record from our code should we give relations a bit more structure,
  and, prove that every object in this new (more structured) world is a mere proposition.
  One good option would be to encode a universe $U$ of mere propositions and have relations
  defined as $B \rightarrow A \rightarrow U$. This extra structure allows us to prove proof irrelevance
  for all $u : U$ (which holds by construction, once $u$ is a mere proposition), but only lets one define
  relations over $U$, which is less expressive than $Set$. The additional boilerplate code is also big,
  once we have to define a language and all operations that we'll need to perform with it. 
  Our universe is defined as:
  
  \Agda{Universes}{data-U} 
  
  With its interpretation back to Agda \D{Set}s beeing:
   
  \Agda{Universes}{data-sharp}
  
  Notation $\mid\mid\_\mid\mid$ denotes propositional truncation. That is, for every type $A$,
  there is a type $\mid\mid A \mid\mid$ with two constructors: (i) for any $x : A$ we have
  $\mid x \mid : \mid\mid A \mid\mid$; (ii) for any $x , y : \mid\mid A \mid\mid$, we have
  $x \equiv y$. This is also called \emph{smashing} sometimes. Not every type constructor 
  preserves mere propositions. A simple example is the coproduct
  $1 + 1$ itself. Even though $1$ is a mere proposition, $1 + 1$ is not, since the elements of such type
  contain also information about which injection was used. Thus the need to smash this information out
  if we want to keep with mere propositions.
  
  It turns out that we are just defining the logic we will need to define relations, but this
  structure is very helpful! Now we can prove that given $u : U$, $\sharp u$ is a mere proposition.
  
  \Agda{Universes}{uprop-decl}
  
  So far so good! We just removed one instance from the user and proving decidability becomes
  straightforward (but in a few, rare, complicated cases)! However, the changes were not only for the better.
  A new, minor, problem is the verbosity introduced by $U$, since everything is harder to write and read.
  But there is a more serious situation happening here. If we look at the existential quantifier defined in $U$,
  its witnesses must also come from $U$. This can be very restrictive once we start using relational
  composition (which is defined in terms of existentials).

\subsubsection{The Equality Model}
  
  Given the options discussed above, with all their positive and negative aspects, it seems
  a little too restrictive to adopt only one option. We indeed mixed both the $\releq$ promote
  with $\subseteq\hyph antisym$. The idea is that users not only can chose how formal they want their model
  to be, but this can significantly increase development speed. In the first stages of development, where
  the base relations might change a lot (and, if instances were written, they would consequently change),
  one can keep developing with the $\releq$ promotion. Once this model is stable, it can completely
  formalizeb by adding the desired instances and using subrelation anti-symmetry.

\section{Constructions}

After establishing a model for relations and relational equality, we follow by presenting
some of the important constructions here. Note that contrary to \emph{pen and paper} Mathematics,
we provide an encoding of the constructions and then we prove that our encoding satisfies the
universal property for the given construction, instead of adopting such property as the
definition. This not only proves the encoding to be correct, but it is the only constructive approach
we can use.

The design adopted for the lower level constructions has a somewhat heavy notation. The
main reason for this choice (which differs significantly from other Relational Algebra implementations)
is its ease of use when coupled with reflection techniques. If we provide all definitions
as Agda functions, when we access a term AST, Agda will normalize and expand such definitions.
By encapsulating it in records, we can stop this normalization process and use a (much) smaller
AST representation.


\subsection{Composition}

Given two relations $R : B \rightarrow C$ and $S : A \rightarrow B$, we can construct a
relation $R \cdot S : A \rightarrow C$, read as $R$ after $S$, and defined by:
\[
  R \cdot S = \{ (a, c) \in A \times C\;\mid\; \exists b \in B\;.\; a\;S\;b \wedge b\;R\;c \}
\]
Or, using diagrams:
\begin{displaymath}
  \xymatrix{ A \ar[r]^S \ar@/_1pc/[rr]_{R \cdot S} & B \ar[r]^R & C }
\end{displaymath}

As a first definition in Agda, one would expect something like:

\Agda{RelationsCore}{composition-naive}

And here we can start to see the dependent types shining. The existential quantification
is just some syntax sugar for a dependent product. Therefore, for constructing a composition
we need to provide a witness of type $B$ and a proof that $c R b \wedge b S a$, given $c$ and $a$.

Yet, this suffers from the problem mentioned earlier. Agda will normalize every occurrence of \F{\_after\_}
to its right-hand side, which is a non-linear lambda abstraction, and will make subterm matching 
very complex to handle. An option is to use the exact definition of an existential quantifier,
but expand it:

\Agda{RelationsCore}{composition-final}


\subsection{Products}
\label{sec:relations:products}

Given two relations $R : A \rightarrow B$ and $S : A \rightarrow C$, we can construct a relation
$\langle R , S \rangle : A \rightarrow B \times C$ such that $(b,c)\;\langle R , S \rangle\;a$ if and only if $b\;R\;a \wedge c\;S\;a$.
Without getting in too much detail of what it means to be a product, we usually write
it in the form of a commutative diagram:

\begin{displaymath}
\xymatrix{%
B & B \times C \ar[l]_{\pi_1} \ar[r]^{\pi_2} & C \\
  &     A \ar@{..>}[u]|{\langle R , S \rangle} \ar[ul]^{R} \ar[ur]_{S}
  &
}
\end{displaymath} 

That is,
\begin{eqnarray*}
  R &=& \pi_1 \cdot \langle R , S \rangle \\
  S &=& \pi_2 \cdot \langle R , S \rangle
\end{eqnarray*}

Products are unique up to isomorphism, which explains the notation without introducing any new names.
The proof is fairly simple and can be conducted by \emph{gluing} two product diagrams.
The diagrammatic notation states the existence of a relation $\langle R , S \rangle$ and the dotted arrow
states its uniqueness. $\pi_1\;(b , c) = b$ and $\pi_2\;(b , c) = c$ are the canonical
projections.\\

\begin{mydef}[Split Universal]
Given $R$ and $S$ as above, let $X : A \rightarrow B \times C$, then:
\[
  X \subseteq \langle R , S \rangle \Leftrightarrow \pi_1 \cdot X \subseteq R \wedge \pi_2 \cdot X \subseteq S
\]\\
\end{mydef}
Encoding this in Agda is fairly simple, once we already have products (in their categorical sense)
available. We just wrap everything inside a record:

\Agda{RelationsCore}{product-final}

Its universal property can be derived from the following \emph{lemmas}

\Agda{RelationsCore}{product-univ-r1}\\
\Agda{RelationsCore}{product-univ-r2}\\
\Agda{RelationsCore}{product-univ-l}

In fact, the product of relations respects both decidability and propositionality (that is,
given two mere propositions, their product is still a mere proposition). Therefore, such 
instances are already defined.

\subsection{Coproduct}

If we flip every arrow in the diagram for products, we arrive at a dual notion, usually
called coproduct or sum. Given two relations $R : B \rightarrow A$ and $S : C \rightarrow A$,
we can perform a \emph{case analysis} in an element of type $B + C$ and relate it to an $A$.
Indeed, the \emph{either} of $R$ and $S$, denoted $[R , S]$, has type $B + C \rightarrow A$
and is depicted in the following diagram:

\begin{displaymath}
\xymatrix{%
 B \ar[r]^{\iota_1} \ar[dr]_{R} & B + C \ar@{..>}[d]|{[R , S]} & C \ar[l]_{\iota_2} \ar[dl]^{S} \\
   &   A   &
}
\end{displaymath}

In Agda, we have:

\Agda{RelationsCore}{coproduct-final}\\
\Agda{RelationsCore}{coproduct-uni-r1}\\
\Agda{RelationsCore}{coproduct-uni-r2}\\
\Agda{RelationsCore}{coproduct-uni-l}

The coproduct has instances already defined for decidability and mere propositionality.

\subsection{Relators and Catamorphisms}

Our library already handles a nice portion of Relational Algebra but one of the most
useful operations, induction! Relational catamorphisms allows one to define relations
by induction over some specified structure. 

Catamorphisms are defined as the \emph{least} fixed points of certain recursion equations. Their
dual are Anamorphisms, which are defined as \emph{greatest} fixed points. There recursion equations
are modelled by recursive functors (which must also be Relators, in the relational setting).\\

\begin{mydef}[Relator]
Let $F$ be a monotonic endofunctor in \catname{Rel}. We say that $F$ is a \emph{relator}
iff, if $R \subseteq S$, then $F\;R \subseteq F\;S$.
\end{mydef}

It can be proved that relators also preserve converse, or, an even more interesting result: a functor
is a relator if and only if it preserves converses, theorem 5.1 in \cite{Bird97}. Unfortunately,
in Agda things are not so simple. The best option is to encode the relator laws as a typeclass.

\Agda{Catas}{relator}

Yet, the purpose of this dissertation is not to explain Relators or Functors, but how to encode them
in Agda. Going one step at a time, let us look at the tradicional definition of a catamorphism.

Categorically speaking, let $F : \catname{C} \rightarrow \catname{C}$ be an endofunctor with
initial algebra $(\mu F , in_F)$, then, for any object $B$ and F-algebra $R : F B \rightarrow B$
the following diagram commutes.

\begin{displaymath}
\xymatrix{%
  \mu F \ar@/^/[r]^{in_F^\circ} \ar@{..>}[d]_{\scata{R}}
     & F (\mu F) \ar@/^/[l]^{in_F} \ar[d]^{F \scata{R}} \\
  B     & \ar[l]_{R} F B
}
\end{displaymath}

\emph{Reading} the universal depicted above, we have:
\[ \scata{R} \equiv_r R \cdot F\;\scata{R} \cdot in_F^\circ \]

Although it looks (and it indeed is) pretty tricky to encode this directly into Agda, at least it gives
us a type to follow. Since almost all usefull functors are polymorphic, we will now
denote $F$ by $F_A$, and regard $F$ as a functor with type $\catname{D} \times \catname{C} \rightarrow \catname{C}$.

We seek to define a relation $\scata R : \mu F_A \rightarrow B$, given $R : F_A B \rightarrow B$,
or, with Agda types, given a $B \rightarrow F_A B \rightarrow \D{Set}$, we need to build a $B \rightarrow \mu F_A \rightarrow \D{Set}$.

\subsubsection{Shape and Positioning}

The problem's core lies in how to define a fixed point in Agda, which turns out 
to be very complicated, if not impossible\footnote{%
%%%% BEGIN FOOTNOTE
With the help of Coinduction one can define fixed points, but this triggers some other
problems when using such fixed points. Not to mention the instability of the Coinduction module.
%%%% END FOOTNOTE
}. One solution is to look at a functor from a different angle. Playing around with
the List functor we arrive at:

\begin{eqnarray*}
  F_A X & \cong & 1 + A \times X \\
        & \cong & 1 \times 1 + A \times X \\
        & \cong & 1 \times X^\bot + A \times X^1
\end{eqnarray*}

The last step of the above derivation, were we transform $1$ into $X^\bot$, can not be proven
in Agda. However, it is evident to see it is a valid transformation from the initiality of $\bot$.

It is important to note that the resulting term is an instance of a more general form: 
\[ T_{S, P} X = \Sigma\; (s : S)\; (X^{P\; s})\] 
For $S = 1 + A$ and $P = [ \text{const }\bot , \text{const }1 ]$. It can be proved
that all polynomial functors can be encoded in that fashion\cite{Abbott04}, where $S$ represents
a set of constructors of our \emph{datatype} and $P$ represents the arity of each constructor.
In the List example, $nil = \IC{i1\; unit}$ has arity 0 and $cons = \IC{i2 (x, \_)}$ receives
one recursive argument.

\Agda{Catas}{wfunctor}

In the excerpt above we have a polymorphic shape $S$, this parameter
only represents the type our container is storing. For the sake of simplification we will forget about it.
The set $W_S P$ is the set of all \emph{well founded trees} constructed by regarding each $s : S$ 
as a constructor of arity $P\; s$. The first definition of W-types is from \cite{lof84}.

\Agda{Catas}{wtype-def}

The important question being: are we talking about \emph{the same} lists? And the answer 
is yes. Abbot et al. did a lot of work on container types
and encoding inductive types as W-types. The most relevant result for us can be found in \cite{Abbott04} and
says that $W_S P$ coincide with the least fixed point of a functor identified by $S$ and $P$:
\[
  W_S P \cong \mu X\; . \;T_{S , P} X
\]

The elimination rule for W-types is

\Agda{Catas}{w-elim}

which gives out the generic induction principle for W-types. The dependent type setting, however, allows
one to tweak the conclusion $C$, depending on the $W_S P$ being \emph{folded}. If we let
$C = A \rightarrow \D{Set}$, quantified on $A$, we arrive at something very close to:

\Agda{Catas}{w-elim-rel}

Where its parameter function $h$ receives the \emph{shape} and \emph{positioning} of a list $l$
(that is, the $in^\circ\;l$), a relation built recursively, and something of type $A$.
The universal law can provide some interesting intuition. It says that $\scata{R}$ is
the same as opening up the list, checking if the recursive parts \emph{relate} with something,
then relating the head and that something with $R$. That is exactly how we encode it:

\Agda{Catas}{w-cata}

Now we just need to wrap it over a record to prevent evaluation, but looking at the type
of \F{W-cata-F}, it is already the type of a relational catamorphism. The final piece of the puzzle looks like:

\Agda{Catas}{cata}

Wrapping it up, in order to handle arbitrary\footnote{
%%%%%% BEGIN FOOTNOTE
Not at all that aribtrary, we can just handle strictly positice inductive types, of which
the polynomial functors are part of.
%%%%%% END FOOTNOTE
} functors the idea is to: (a) encode it using W-types; (b) Use a custom elimination rule, derivable
from the standard W-elimination and (c) translate the relational gene using the cata universal. 
A small problem with this technique is the need to explicetly prove the cata universal for
every functor we wish to use. One way of bypassing this would be to postulate it. Since
we mainly need the universal for rewriting purposes, no actual computaton needs to be performed.

Proving the universal law for a specific functor is straigh-forward. Nevertheless, we adopted
the postulate strategy to ease the development. Remember that in the relational case, the universal
law is split in two separate laws:

\Agda{Catas}{cata-uni}


\subsection{Properties}

Besides the main relational constructions, we developed a selection of property-specific modules.
Given the amount of properties we have in relational algebra, we had to somehow categorize them.
Modules are divided in the kind of property they provide. The naming scheme is straigh forward:
names start with a hyphen separated list of relevant operators, followed by the property name.
If we are looking for $\cdot + \cdot$ and $(\cdot)^\circ$ distributivity, for instance, 
its name is going to be $+$-$^\circ$-distr. An aditional identification: -$l$, -$r$, 1 or 2; might be found
appended to the property name, indicating either the direction (left or right) or the part of 
the conclusion being extracted. One example is the split universal, section \ref{sec:relations:products}.
A simple taxonomy of modules is provided below.

\begin{enumerate}
  \item \emph{Group} like properties, such as: neutral elements, absorption, inverse and associativity.
  \item \emph{Monotonicity} over $\_\subseteq\_$.
  \item \emph{Idempotence} of operators.
  \item \emph{BiFunctoriality} for product and coproduct.
  \item \emph{Correflexive} specific properties.
  \item \emph{Galois} connections present in Relational Algebra, which provide
                distributivity over $\_\cap\_$ and $\_\cup\_$, cancelation and
                more monotonicity.
\end{enumerate}

\section{The Library}

The constructions and workarounds explained above do not constitute a full description of the library.
There are still modules for equational reasoning and a few more constructs implemented.
All the code is available in the following GitHub repository: 
\begin{quote}
\small https://github.com/VictorCMiraldo/msc-agda-tactics/tree/master/Agda/Rel
\end{quote}
It is compliant with Agda's version 2.4.2.2; standard library version 0.9.

\subsection{Summary}

This chapter discussed, somewhat high level, the implementation and problems
we found during the development of our Relational Algebra library. Even though some features
might seem very complex, it is the price we have to pay in exchange for the expressiveness 
of dependent types. 

The problem imposed by Agda's (intensional) equality is that it implies
convertibility. As seen in this chapter, we don't really care about convertibility of relations.
They need to be inhabited at the same time, though. 
Borrowing some notions from Homotopy Type Theory\cite{hottbook}
in order to hardcode a proof irrelevance notion was very helpful, besides the complex layer of boilerplate
code for the end-user. The later option, postulating some axioms that allows one to \emph{trick}
Agda's equality, revealed itself to be cleaner and to provide both a better interface
and a completely formal approach, when needed.

Another complicated solution was the use of W-types to encode generic (polynomial) functors,
with their respective catamorphism. Although some mechanical code had to be pushed to the user
we finished with an interface very close to what one would write on paper, which closes
the features we were looking forward to have.


