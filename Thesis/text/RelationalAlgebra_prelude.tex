This work takes Relational Algebra\cite{Bird97} as main case study for a couple of reasons. One of the most
important is its expressive power and unquestionable advantage as a framework for
reasoning about software, in an equational fashion. Relational Algebra is in fact a discipline that
allows us to speak of software engineering through unambiguous equations, giving a definite meaning
to the \emph{engineering} part of software engineering.

We are not the first to use Agda and Relational Algebra together, though. There are two approaches
that deserve to be mentioned and compared to what we are doing here. The first, which in fact
should be regarded as a basis for our work, is due to Mu et al, at \cite{Jansson09}. 
The second, which is more abstract, is due to Kahl, at \cite{RATHAgda}.

This chapter will present and compare both approaches mentioned above, 
then introduce a basic encoding for relations. This will be followed by the discussion of several 
further options that would change the handling of equality, which turned out to be a very
delicate matter. 
