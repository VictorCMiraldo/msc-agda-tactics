\begin{TODO}
  \item frame this thing
\end{TODO}

We're taking Relational Algebra\cite{Bird97} as our main case study for a couple of reasons. One of the most
important beeing it's expressive power and it's unquestionable advantage as a framework for
reasoning about software, in an equational fashion. Relational Algebra is in fact a discipline that
allows us to speak of software engineering through unambiguous equations, giving a definite meaning
to the \emph{engineering} part of software engineering.

In this chapter we will introduce a basic encoding for relations, then we'll discuss several 
further options that would change the handling of equality, which turned out to be a very
delicate matter. The problem imposed by Agda's (intensional) equality is that it implies
convertibility. As we shall see later, we don't really care about convertibility of relations,
but that they're inhabited at the same time. We can borrow some notions from Homotopy Type Theory\cite{hottbook}
in order to hardcode a proof irrelevance notion, but this adds a very complex layer of boilerplate
code for the end-user. Another option is to postulate some axioms that will allow us to \emph{trick}
Agda's equality, beeing always carefull not to introduce a contradiction. 
There are a lot of options for this later case too, we shall discuss a few of them.

\section{Encoding Relations}

Unlike most programming languages, an encoding of Relational Algebra in a dependently typed 
language allows one to truly see the advantage of dependent types in action. Most of the definitions
happens at the \emph{type level} (remember that there is no difference between types and values in Agda,
this is just a mnemonic to help understanting the \emph{functions}). The encoding presented
here is based on \cite{Jansson09}.

Long story short, a binary relation $R$ of type $A \rightarrow B$ can be thought of in terms of several mathematical objects.
The usual definition is to say that $R \subseteq A \times B$, where $\cdot\times\cdot$ is the cartesian product of sets.
In fact, $R$ contains pairs or related elements whose first component is of type $A$ and second component is of type $B$,
note that it is slightly more general than the concept of a function, where we can have $b_1\;R\;a$ and $b_2\;R\;a$, which means
that $(a, b_1) \in R$ and $(a, b_2) \in R$, as a perfectly valid relation, but not a function, since $a$ would be mapped to two different values, $b_1$ and $b_2$.

\newcommand{\powerset}{\mathcal{P}}
Another way of speaking about relations, though, is to consider functions of type $A \rightarrow \powerset B$.
If our previous $R$ was in fact a function $f$, we would then have $f\; a = \{b_1, b_2\}$. For the more
mathematically inclined reader, the arrows $A \rightarrow \powerset B$ in the category \catname{Sets}, of sets and
functions, correspond to arrows $A \rightarrow B$ in the category \catname{Rel}, of sets and relations. For this matter,
we actually call \catname{Rel} the Kleisli Category for the monad $\powerset$. 
In fact, we can even define the powerset transpose of a relation $R$, which is a function that for each $a$
returns the subset of $B$ that is related to $a$ trough $R$.
\[ (\Lambda\; R)\; a = \{ b \in B \;\mid\; b\;R\;a \} \]

For our Agda encoding of Relations, we shall use a slight modification of the $\powerset$ approach.
Let's begin, in fact, by encoding set theoretic notions. The most important of all beeing undoubtly
the membership notion $\cdot \in \cdot$. One way of encoding a subset of a set $A$ is using a function $f$
of type $A \rightarrow Bool$, the subset is obtained by $\{\; a \in A\; |\; f\;a\; \}$. Yet, in Agda, this would
force that we deal only with decidable domains, which is not a problem if everything is finite, but
for infinite domains this would not work.

Another option, which is the one used in \cite{Jansson09}, is to use a function $f$
of type $A \rightarrow \D{Set}$ to encode a subset of $A$. Remember that \D{Set} is
the type of types in Agda. Although not very intuitive, this is much more expressive than the 
last option and the induced subset would be defined by $\{\; a \in A\; | f\;a\;\text{is inhabited }\}$,
which would turn out to be:

\Agda{RelationsCore}{subset-def}

Extending from sets to binary relations is a very simple task. Besides the canonical steps,
we'll also swap the arguments for a relation, following what was done in \cite{Jansson09} and
keeping the syntax closer to what one would write on paper, since we usually writes a relational
statement from left to right, that is, $y\;R\;x$ means that $(x, y) \in R$.
\begin{eqnarray*}
  \powerset (A \times B) & = & \powerset (B \times A)\\
                         & = & B \times A \rightarrow \D{Set}\\
                         & = & B \rightarrow A \rightarrow \D{Set}
\end{eqnarray*}

Here we present the the base encoding for relations in Agda, together with a few constructs
to help in their definition and to illustrate usage.

\Agda{RelationsCore}{relation-def}

And the operations are defined just as we would expect them. Note that \D{\_⊎\_} represents a disjoin union of sets
(equivalent to Either in Haskell) and \D{\_×\_} represents the usual cartesian product. The function
lifting might look like the less intuitive construction, but it's a very simple one. $fun\;f$ is a relation,
therefore it takes a $b$ and a $a$ and should return a type that is inhabited if and only if $(a, b) \in fun\;f$,
or, to put it another way, if $b = f\;a$, almost like it's textbook definition.

\begin{TODO}
  \item Fix the definition of composition in the library. Explain it here.
  \item Add a few universals. Explain how we're taking the road in the "wrongway".
        We first give a definition then we prove it satisfies the universals, instead
        of using universals as definitions.
  \item Explain Agda's propositional equality?
\end{TODO}

\section{Relational Equality}

A notion of convertibility is paramount to any form of rewriting. In out scenario, whenever
two relations are equal, we can substitute them back and forth in a given equation. There are
a couple different, but equivalent notions of equality. The simplest one is borrowed from set
theory.\\

\begin{mydef}[Relation Inclusion]
Let $R$ and $S$ be suitably typed binary relations, we say that $R \subseteq S$ 
if and only if for all $a , b$, if $b\;R\;a$ then $b\;S\;a$.\\
\end{mydef}

\begin{mydef}[Relational Equality]
Let $R$ and $S$ be as before, we say that $R \releq S$ if and only if
$R \subseteq S$ and $S \subseteq R$.\\
\end{mydef}

\begin{lemma}
Relation Inclusion is a partial order, that is, $\subseteq$ is reflexive, transitive and anti-symmetric.
And $\releq$ is an equivalence relation (reflexive, transitive and symmetric).\\
\end{lemma}

So far so good, at first sight one would believe that there are no problems with such definitions
(and in fact there won't be any problem as long as we keep away from Agda, but then it wouldn't be fun, right?). 
Ignore the boilerplate code, we wraped both definitions
as datatypes in Agda to prevent expansion when using reflection.

\Agda{RelationsCore}{subrelation}

Allow us to take a deeper look into what this definition say, though. 
Let $R : \D{Rel}\;A\;B$, for $A$ and $B$ \D{Set}s, $a : A$ and $b : B$. When we
state $R\;b\;a$ we're obtaining a \D{Set}, and an element $r : R\;b\;a$ represents
a proof that $b$ is related to $a$ trough $R$. We have no interest in the contents of
such proof though, it's existence is what matters. In more formal terms, we would like
that $\D{Rel}$ returned a proof irrelevant set. In Coq this would be \texttt{\small Prop}, but
Agda has no set of propositions.

We can see such problem comming to the surface if we try to use Agda's propositional equality 
instead of $\releq$ (and hence the name distinction):

\Agda{RelationsCore}{releq-is-not-propeq}

We have to use functional extensionality anyway (here, disguised as \F{rel-ext}). But the 
type of our goal is $R\;b\;a \equiv S\;b\;a$ and the proof get stuck since set equality is
undecidable \warnme{is it? Be sure!}. We infact just want $R$ and $S$ to be inhabited at
the same time.

So, it's clear that we also need some encoding of $\releq$:

\Agda{RelationsCore}{releq}

But this definition is not substitutive in Agda. Therefore we need to lift it to
a substitutive definition. In fact, it is safe to lift it to the standard propositional equality.
The advantages of doing so are that we win all of the functionality to work with $\equiv$ for free.

The solution we're currently adopting relies on using yet another disguised version of
function extensionality (here as powerset transpose extensionality) 
and postulating the proof-irrelevant notion of set equality (as in Set Theory):

\Agda{RelationsCore}{releq-lifting}

The powerset transpose postulate is pretty much function-extensionality with some syntatic sugar, and it is
known to not introduce any contractiction. But $ℙ-ext$ on the other hand, have to be justified.
The reason for such postulate is just because somewhere in our library, we gotta have a notion
of proof-irrelevance to be able to state relational equality. As we'll discuss in the next section,
there are a couple places were we can insert this notion, but postulating it at the subsets ($ℙ$) level is
by far the easier to handle and introduces the least amount of boilerplate code. (Remember that
if $a$ is a subset of $A$, that is $a : ℙ A$, stating $a\;x$ is stating $x \in a$, for any $x : A$.).

There are a few other options of achieving a substitutive behavior for $\releq$, they all
rely on it's lifting to propositional equality, though. As we shall present in the next 
sections.

\begin{TODO}
  \item Indirect equality somewhere?
\end{TODO}

\subsection{Hardcoded Proof Irrelevance}

  We mentioned before this idea of a proof irrelevant set, we also mentioned that we postulated
  such notion, which is not the ideal thing to do in Agda. We want to keep our postulates to an
  absolute minimum.
  
  In order to formally talk about proof irrelevance we need some concepts from Homotopy Type Theory\cite{hottbook}.
  HoTT is an immense and complex newly founded field of Mathematics, and we are not going to explain
  it in detail. The general idea is to use abstract Homotopy Theory to interpret types. In normal
  type theory, a statement $a : A$ is interpreted as $a$ is an element of the set $A$. In HoTT, however,
  we say that $a$ is a point in space $A$. Objects are seen as points in a space and the type $a = b$ is seen as a path from $a$ to $b$.
  It's obvious that for all $a$ there is a path from $a$ to $a$, therefore $a = a$, giving us reflexivity.
  If we have a path from $a$ to $b$, we also have one from $b$ to $a$, resulting in symmetry.
  \emph{Glueing} of paths together corresponds to transitivity. So we have an equivalence relation $=$.
  
  Even more important, if we have a proof of $P a$ and $a = b$, for a predicate $P$, we can
  transport this proof along the path $a = b$ and arrive at a proof of $P b$. This corresponds to
  the substitutive behavior of standard equality. Note though, that in classical mathematics,
  once an equality $x = y$ has been proved, we can just switch $x$ and $y$ whenever we want. 
  In HoTT, however, there might be more than one path from $x$ to $y$, and they might yield different
  results, it is important to state which path is beeing \emph{walked} when we use substitutivity.
  
  The first important notion is that of homotopy. Traditionally, we take that two functions are
  the same if they agree onall inputs. A homotopy between two functions is very close to that
  classical notion:
  
  \Agda{ProofIrrel}{homotopy-def}
  
  It's worth mentioning that \F{\_\~\_} is an equivalence relation, although we ommit the proof.
  We follow by defining an equivalence, which can be seen an isomorphism notion:
  
  \Agda{ProofIrrel}{isequiv-def}
  
  That is, $f$ is an equivalence if there exists a $g$ such that $g$ is a left and right-inverse
  of $f$.
  
  Now we can state when two types are equivalent. And for the reader familiar with algebra, it is
  not very far from the usual isomorphism-based equivalence notion. As expected, univalence is
  also an equivalence relation.
  
  \Agda{ProofIrrel}{univalence-def}
  
  Following the common pracice when encoding HoTT in Agda, we have to postulate the Univalence
  axiom, which in short says that univalence and equivalence coincide:
  
  \Agda{ProofIrrel}{univalence-axiom}
  
  Now, our job becomes much easier, and it suffices to show that if two relations are
  mutually included, then they are univalent.
  
  \subsubsection{Mere Propositions}
  
  As we mentioned earlier, proof irrelevance is a desired property in most systems. In HoTT,
  one distinguish between mere propositions and other types, where mere propositions
  are defined by:
  
  \Agda{ProofIrrel}{isprop-def}
  
  This allows us to state some very usefull properties, which allows us to handle propositions
  as both true or false, depending on wether or not they're inhabited. These corresponds to
  lemma 3.3.2 in \cite{hottbook}.
  
  \Agda{ProofIrrel}{lemma-332}\\
  \Agda{ProofIrrel}{not-lemma-332}
  
  Which are both provable in Agda, but the proofs are ommited here.
  
  \subsubsection{Adding Relations to the mix}
  
  Now, to exploit such finer treatment of equality in our favor, we need to add
  relation and a few other details to the mix. We'll keep the relation definition
  as before, and push to the user the responsability of proving that his relations
  are both mere propositions and decidable.
  
  This can be easily done with Agda's instance mechanism:
  
  \Agda{ProofIrrel}{instances}
  
  This will treat both records as typeclasses in the Haskell sense. Now, for talking about
  subrelations they must be an instance of \D{IsProp}, and whenever we wish to use anti-symmetry
  they must also be an instance of \D{IsDec}, and it turns out that anti-symmetry is now
  provable as long as we assume relational extensionality.
  
  \Agda{ProofIrrel}{subrel-def}\\
  \Agda{ProofIrrel}{subrel-antisym}
  
  Although we could arrive at the result we desired with a minimal number of postulates (relational extensionality
  and univalence axiom), the user was heavily punished for when he wants to define a relation, not to
  say that decidability will give problems once relational composition enters the stage. For this reasons
  we chose not to adopt this solution, even though it's more formal than what we have.

\subsubsection{Custom Universes}


  We could remove the \D{IsProp} record from our code, if we gave relations a bit more structure,
  and, prove that every object in this new (more structured) world is a mere proposition.
  One good option would be to encode a universe $U$ of mere propositions and have relations
  defined as $B \rightarrow A \rightarrow U$. This extra structure allows us to prove proof irrelevance
  for all $u : U$ (which holds by construction, once $u$ is a mere proposition), but only let's us define
  relations over $U$, which is less expressive than $Set$. The additional code boilerplate is also big,
  once we have to define a language and all operations that we'll need to perform with it.
  
  A universe and it's interpretation are defined as:
  
  \Agda{Universes}{data-U}
  
  Where $\mid\mid\_\mid\mid$ is a propositional truncation. That is, for every type $A$,
  there is a type $\mid\mid A \mid\mid$ with two constructors: (i) for any $x : A$ we have
  $\mid x \mid : \mid\mid A \mid\mid$; (ii) for any $x , y : \mid\mid A \mid\mid$, we have
  $x \equiv y$. This is also called smashing sometimes. Not every type constructor 
  preserves mere propositions. A simple example is with the coproduct
  $1 + 1$. Even though $1$ is a mere proposition, $1 + 1$ is not, since the elements of such type
  contain also information about which injection was used; We need to smash this information out
  if we want to keep with mere propositions.
  
  It turns out that we're just defining the logic we'll need to define relations, but this
  structure is very helpful! Now we can prove that given $u : U$, $\sharp u$ is a mere proposition.
  
  \Agda{Universes}{uprop-decl}
  
  So far so good! We just removed one instance from the user and proving decidability can be very
  straight forward (but in a few, rare, complicated cases)! Well, the changes were not only for the best.
  A new, minor, problem is the verbosity introduced by $U$, everything is harder to write and read.
  But there's a bigger situation happening here. If we look at the existential quantifier defined in $U$,
  it's witnesses must also come from $U$. This can be very restrictive once we start using relational
  composition (which is defined in terms of an existential).


