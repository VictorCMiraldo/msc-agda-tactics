We're taking Relational Algebra\cite{Bird97} as our main case study for a couple of reasons. One of the most
important being it's expressive power and it's unquestionable advantage as a framework for
reasoning about software, in an equational fashion. Relational Algebra is in fact a discipline that
allows us to speak of software engineering through unambiguous equations, giving a definite meaning
to the \emph{engineering} part of software engineering.

We are not the first to use Agda and Relational Algebra together, though. There are two approaches
that deserve to be mentioned and compared to what we are doing here. The first, which is the main
basis for our work, is due to Mu et al, at \cite{Jansson09}. The second, which is more abstract,
is due to Kahl, at \cite{RATHAgda}.

In this chapter we will present both approaches mentioned above, 
then introduce a basic encoding for relations. We'll follow with the discussion of several 
further options that would change the handling of equality, which turned out to be a very
delicate matter. The problem imposed by Agda's (intensional) equality is that it implies
convertibility. As we shall see later, we don't really care about convertibility of relations,
but that they're inhabited at the same time. We can borrow some notions from Homotopy Type Theory\cite{hottbook}
in order to hardcode a proof irrelevance notion, but this adds a very complex layer of boilerplate
code for the end-user. Another option is to postulate some axioms that will allow us to \emph{trick}
Agda's equality, being always careful not to introduce a contradiction. 
There are a lot of options for this later case too, we shall discuss a few of them.
