This document starts with a basic mathematical backgroung, in chapter \ref{chap:background}, where we try to help the unfamiliar reader to understand
how programming and proving are intrinsecally connected. These ideas
are very important to have a solid understanding of the Agda language,
which we introduced, shallowly, in section \ref{sec:prelude:agdalanguage}.

We follow with the description of our Relational Algebra library, chapter \ref{chap:relationalalgebrainagda}, where we explore the full expresivity of 
dependent types to encode relations in such a way that they remain
usefull for automatic proving. The library is by no means complete and
should be taken as an exploration of what is possible or not. The code
behaves very well until we start to use generic catamorphisms.

Chapter \ref{chap:termsandrewritting} will focus on rewriting. The context
in which we immerse ourselfes is given by sections \ref{sec:tandr:reasoning} and
\ref{sec:tandr:reflection}, the rest of the capter is concerned with
explaining the basic form of rewriting one can automate. Generalizations
on this rewriting framework follows on chapter \ref{chap:variations}.

On chapter \ref{chap:futurework} we give pointers for future work and
conclude the work.

