\begin{TODO}
  \item transitivity + mixfix = eq reasoning
\end{TODO}

The possiblity of defining mixfix operators makes Agda a very customizable language. This is,
indeed, used very often to build domain specific languages on top of Agda itself, which 
gives a very convenient way of proving results. 

By taking a look at a piece of squiggol, from a type-theoretic angle, one can identify
a few components. As an example, let us take a proof of the shunting property, from \cite{Bird97},
and look at its syntax.

\begin{eqnarray}
&&
  f \cdot R \subseteq S
\just{\Leftarrow}{$f$ is simple}
  f \cdot R \subseteq f \cdot f^\circ \cdot S
\just{\Leftarrow}{monotonicity$_2$}  
  R \subseteq f^\circ \cdot S
\justl{\Leftarrow}{just:type1}{$f$ is entire}
  f^\circ \cdot f \cdot R \subseteq f^\circ \cdot S
\justl{\Leftarrow}{just:type2}{monotonicity$_1$} 
  f \cdot R \subseteq S \nonumber
\end{eqnarray}

The first thing that pops up is the implication, $\Leftarrow$, which is not part of the relational
calculus being developed in the subject of the demonstration. The justifications, however,
must have type $e_{n+1} \Leftarrow e_{n}$. Well, implication in Agda is just the arrow type, as for functions,
rendering the type of the justifications to be $e_n \rightarrow e_{n+1}$. Since we have loads of justifications,
it makes sense to associate them to some side. By assuming they are right-associative we arrive at
a ternary mixfix operator $\_\Leftarrow\{\_\}\_$. Transitivity and reflexivity of the underlying
operation, in our case $\Leftarrow$, closes the box.

However, Agda already supports an even more generic approach to equational reasoning. All we need is a preorder relation $\_\sim\_$ with a subjacent equivalence relation $\_\approx\_$. Below we present the kinds of relational reasoning
supported.

\begin{center}
\begin{minipage}[t]{0.3\textwidth}
\begin{eqnarray*}
&& R \\
& \equiv_r & \langle lemma_1 \rangle \\
 && S
\end{eqnarray*}
\end{minipage}%
\begin{minipage}[t]{0.3\textwidth}
\begin{eqnarray*}
&& R \\
& \subseteq & \langle lemma_2 \rangle \\
 && S
\end{eqnarray*}
\end{minipage}%
%
\begin{minipage}[t]{0.3\textwidth}
\begin{eqnarray*}
&& R \subseteq S\\
& \Leftarrow & \langle lemma_3 \rangle \\
 && R' \subseteq S'
\end{eqnarray*}
\end{minipage}
\end{center}

Where the above lemmas must have type $R \equiv_r S$, $R \subseteq S$ and $R' \subseteq S' \rightarrow R \subseteq S$
respectively. The context of equational reasoning is a good starting point to build some intuition about rewriting.

The reader might have noticed a small problem in the above explanation of equational reasoning, 
in contrast with the shunting proof given as an example.
Not every justification in the proof have the same \emph{type}, as we claimed above. Looking at
step \ref{eq:just:type2}, with type $\forall\; R\; S\; T\;.\;R \subseteq S \rightarrow C \cdot R \subseteq C \cdot S$,
and step \ref{eq:just:type1}, with type $id \subseteq f^\circ \cdot f$. Our rewrite framework does exactly that job,
of \emph{adjusting} the lemmas that do not fit. Picture \ref{fig:rewriteillustration} should provide some intuition in that direction.

\begin{figure}[h]
\begin{TODO}
  \item Our nice picture
\end{TODO}
\caption{Rewrite Illustration}
\label{fig:rewriteillustration}
\end{figure}




