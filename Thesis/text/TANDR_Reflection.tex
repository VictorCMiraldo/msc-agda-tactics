As of version 2.4.8, Agda's reflection API provides a few keywords displayed in table
\ref{tbl:agda_reflection_api}. This feature can the thought of as the Template Haskell
approach in Agda, or, metaprogramming in Agda. 

The idea is to access the abstract representation of a term, during compile time,
perform computations over it and return the resulting term before resuming compilation.
We will not delve into much detail on Agda's abstract representation. 
The interested reader should go to Paul van der Walt's thesis\cite{vdWalt2012}, where,
although somewhat outdated, Paul gives a in-depth explanation of reflection in Agda.

We implemented the \emph{by} tactic in agda, which will try to figure out congruences
or substitutions to solve a given goal. The implementation will be discussed below.

\begin{center}
\begin{table}[h]
\begin{tabular}{l l}
  \K{quote} & Returns a \D{Name} datatype, representing the quoted identifier. \\[2mm]
  \K{quoteTerm} & Takes a term, normalizes it and returns it's \D{Term} inhabitant. \\[2mm]
  \K{quoteGoal} g \K{in} f & Brings a quoted version of the goal in place into $f$'s scope,
                          namely, $g$. \\[2mm]
  \K{quoteContext} & Returns a list of quoted types. This is the ordered list of \\
                   & types of the local variables from where the function was called. \\[2mm]
  \K{tactic} f & Is syntax-sugar for $\K{quoteGoal}\;g\;\K{in}\;(f\;\K{quoteContext}\;g)$.
\end{tabular}
\caption{Agda 2.4.8 Reflection API}
\label{tbl:agda_reflection_api}
\end{table}
\end{center}

\begin{TODO}
  \item How much of reflection should I write about?
\end{TODO}
