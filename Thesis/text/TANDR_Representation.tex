\begin{TODO}
  \item glue
\end{TODO} 

The best way to explain how we implemented the \emph{by} tactic is to extensively
work out a few examples. This should provide the justification of some design decisions
and give a tutorial on how to extend our framework.

We are going to focus on two scenarios. The first one is a simple associativity proof
for list concatenation. Whereas the second is a relational proof of a trivial lemma.

\begin{center}
\hspace{-2cm}
\begin{minipage}[t]{0.45\textwidth}
\Agda{RTerm-Examples}{example1}
\end{minipage}
\begin{minipage}[t]{0.45\textwidth}
\Agda{RTerm-Examples}{example2}
\end{minipage}
\end{center}

One could fill in the holes manually, with the terms presented in table \ref{tbl:example_solutions}.
Our task is to generate such terms automatically from the information supplied by Agda: the goal type and
context list\footnote{list of types, in the same order as the De Bruijn indexes of the function parameters};
together with the action given by the user (identified by orange text in table \ref{tbl:example_solutions}).

\begin{table}[h]
\begin{tabular}{c c l}
  Hole 0 & $ = $ & $ \F{cong}\; (\IC{\_∷\_}\; x)\; (\K{\textit{++-assoc}}\; xs\; ys\; zs) $ \\
  Hole 1 & $ = $ & $ \F{subst}\; (\lambda\; x\; →\; R\; ⊆\; x)\; (\F{≡r-promote}\; (\K{\textit{∙-id-r}}\; R)) $ 
\end{tabular}
\caption{Solutions for holes 1 and 2}
\label{tbl:example_solutions}
\end{table}

It is already easy to see how different relations will take different solutions. For hole one,
a congruence suffices, since we're dealing with propositional equality and a data constructor (which is always a congruence).
Hole two, on the other hand, needs a partially applied \F{subst}. And this will be the case whenever we're
handling relational equality inside our \emph{squiggol} environment. For the time being, however, let us
forget about this detail.

Figure \ref{fig:goal_act_notation} gives the notation we are going to use to identify subterms in the given
goal and action. We will denote the $n$-th goal by $(hd_n, g_1, g_2)$ and a given action $A$ by $(hd_a, a_1, a_2)$.
Inside the library, this is represented by a \D{RBinApp}, which forces all terms that we propose to handle
to have a binary relation on their heads.

\newcommand{\labelover}[2]{\stackrel{#1 \vspace{1mm}}{#2}}
\begin{figure}[h]
\[
\begin{array}{l c}
    \text{Goal} &
    \underbrace{(x\; ∷\; xs\;\conc\; ys)\; \conc\; zs}_{g_1} \labelover{hd_g}{\equiv} \underbrace{x\; ∷\; xs\;\conc\; (ys\; \conc zs)}_{g_2} \\[0.4cm]
    \text{Action} &
    \underbrace{(xs\;\conc\; ys)\; \conc\; zs}_{a_1} \labelover{hd_a}{\equiv} \underbrace{xs\;\conc\; (ys\; \conc zs)}_{a_2}
\end{array}
\]
\caption{Goal and Action for hole zero}
\label{fig:goal_act_notation}
\end{figure}

An interesting observation is that if we take only the common subterms $g_1$ and $g_2$, we could obtain something like:
\[
  g_\square = g_1 \cap g_2 = (x \; ∷\; \square\; \conc\; \square) \stackrel{lift}{\rightarrow} (x \; ∷\; \square)
\]
Which is very close to the abstraction we are looking forward to derive. The \emph{lift} arrow represents
a lifting of holes, that is, definitions whose arguments are all holes become a single hole.
This small intuition not only gives a simple decidable algorithm for deriving the abstraction from the goal type alone,
but also gives some clues on how to represent terms. The term representation one is looking for has to support \emph{holes}, or, behave as a zipper structure. So far we need a parametrized simplification of Agda's \D{Term}. The representation
we chose is:

\Agda{RTerm}{rtermname-data}\\

\Agda{RTerm}{rterm-data}

Where \D{RTermName} distinguishes from definitions, constructors and function types. Such
distinction is important once we need to convert these back to Agda's \D{Term} before unquoting.
The \D{RTerm} functor is very straight-forward for a Haskell programmer. What might pass by unoticed
is how much more expressiveness one gets from a parametrized type such as \D{RTerm} in agda.
The translation of an Agda term into a \D{RTerm} actually returns a \D{RTerm $\bot$}, from which
we can prove that there are no \IC{ovar} occurences (the actual object of type \D{RTerm $\bot$} \emph{is} the proof).

In fact, we rely heavily on term intersection:

\Agda{RTerm}{rterm-intersect-type}

Where a \IC{just} is used to keep equal \IC{ovar}s and a \IC{nothing} represents a hole, or,
a difference in the arguments. Note that some representations are exactly the same. After getting
a intersection term, with type \D{RTerm (Maybe $\bot$)}, we actually cast it to \D{RTerm Unit}.
They are isomorphic. One could argue that the intersection type could be made simpler, but it
wouldn't be as generic as we wanted our operations to be.

Having defined an intersection, which works somewhat as a term \emph{roadmap}. It makes sense to be
able to fetch the differences in two terms, or, in a term and a \emph{roadmap}:

\Agda{RTerm}{rterm-difference-type}

Which takes two terms, one with holes and a regular one, and if they \emph{match}, return
a list of subterms from it's second argument that overlap the holes in the first one.
As an example:
\begin{eqnarray*}
  (x\;∷\;\square) - (x\;∷\;(xs\;\conc\;ys)\;\conc\;zs) & = & just\; ((xs\;\conc\;ys)\;\conc\;zs\; ∷ \; []) \\[2mm]
  (x\;∷\;\square) - ((xs\;\conc\;ys)\;\conc\;zs) & = & nothing \\[2mm]
  (x\;∷\;\square\conc\square) - (x\;∷\;(xs\;\conc\;ys)\;\conc\;zs)
      & = & just\; (xs\;\conc\;ys ∷ zs ∷ [])
\end{eqnarray*}

From the intersection and difference operators we ca already extract a lot of information to
complete our goal. The general algorithm is as follows:

\begin{enumerate} %[i)]
  \item Given the goal $(hd_g, g_1, g_2)$ and action type $(hd_a, a_1, a_2)$, we start
        by calculating the abstraction $g\square = \uparrow (g_1 \cap g_2)$, where $\uparrow\_$
        represents hole-lifting.
  \item We proceed by separating the parts of the goal which still need to be handled,
        they are $u_1 = g\square - g_1$ and $u_2 = g\square - g_2$.
  \item Unification of $u_1$ with $a_1$ and $u_2$ with $a_2$ follows, this should give two substitutions
        $\sigma_1$ and $\sigma_2$.
  \item Afterwards, it's just a matter of converting all this data back to a term. $\sigma = \sigma_1 \conc \sigma_2$
        will contain the list of arguments we should pass to our action function.
\end{enumerate}

The last step, of converting everything back to Agda's \D{Term} type, should be modular.
A simple idea is to use different strategies triggered by specific $(hd_g,\;hd_a)$ pairs.
This should give us enough freedom to use our library even on very particular scenarios.

\section{Unification}

%\Agda{RTerm}{rterm-data}\\
%\Agda{RTerm}{rtermname-data}\\
%\Agda{RTerm}{rterm-replace}\\
%\Agda{RTerm}{rterm-fmap}\\
%\Agda{RTerm}{rterm-inersect-type}\\
%\Agda{RTerm}{rterm-lift-hole}\\
%\Agda{RTerm}{rterm-intersect-lift}\\
%\Agda{RTerm}{rterm-minus}\\
%\Agda{RTerm}{rterm-minus-single}
