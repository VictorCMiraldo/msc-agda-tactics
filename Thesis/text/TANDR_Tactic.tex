Computing a term that allows a \emph{rewrite}, which is our main task here, is divided in three parts. First we extract data from the goal and action type. We
make sure that these terms have the expected form and satisfy the requirements
imposed on them. The dependent type setting allows us to encode such requirements
at the type-level. This data is called \F{RWData}.

\Agda{Strat}{RWData-def}

Note how we can be sure that both the goal is a closed binary application and the action has the correct number of variables to be instantiated, since $act\mathbb{N}$ represents the action arity. The context is unused so far, but since
it is given by the \K{tactic} keyword we decided to keep it there. To compute such structure one calls:

\Agda{RW}{make-RWData-def}

where \F{Ag2RTypeFin} is defined by:

\Agda{RW}{ag2rtypefin-def}

Which means that we first translate a type normally, for which we get a \F{RTerm $\bot$}, then we lift all \IC{ivar}s there to \IC{ovar}s. \F{R2FinType} finishes up by converting the variables $n$ such that $n \le act\mathbb{N}$ to a finite
representation and downgrades other variables back to \IC{ivar}s. This
gives a term that can be passed directly to instantiation without further processing.

Having computed a \F{RWData}, we need to extract the congruence (resp. substitution) and the parameters, if any, to give to the action. That is, we are now going to compute some \emph{unification data} out of a \F{RWData}.

\Agda{Strat}{UData-def}

Which is composed of a single-hole abstraction, named $\square$. A substitution
for the action parameters, translated to a list of \F{$\mathbb{N} \times \text{RTerm } \bot$} for easier handling and a list of transformations to be applied. Let us postpone this last detail a bit longer. Section \ref{sec:tandr:representingterms} gives a description of how to compute $\square$ and what to instantiate with what afterwards. These could be summarized by the
following unification strategy.

\Agda{Strat}{basic-def}

Some housekeeping functions are present to make sure we perform the correct
conversions. Nevertheless, the statements inside the \K{let} block correspond to the description given in section \ref{sec:tandr:representingterms}. It is
 evident that wrapping everything in an Error monad makes programming much easier. Not only that, but it also allow us to compose strategies in a neat fashion:
 
\Agda{Strat}{runUStrats-def}

Where \F{basic-sym} is the same as \F{basic}, but instantiates the first part of the goal with the second parameter of the action and vice-versa. If such instantiation yields a substitution, we add \emph{symmetry} to the list of
transformations to be applied. In fact, it is the only transformation we support so far. The alternative operator has the same Haskell semantics for
the Error monad.

If we are able to compute a \F{UData} successfully, the last step is to convert
all that information back into a term. This is where customization can take place.

\Agda{Strat}{TStrat-def}

The \F{when} field is a predicate over the goal relation name and the action relation name, in that order, which specifies when \F{how} should be applied.
Whereas \F{how} receives the action name and unification data and should generate a closed term to close the goal in question.

As a simple example, the \F{TStrat} for handling propositional equality is shown below.

\Agda{Strat}{strat-propeq}

\subsection{Interfacing}

One interesting feature of Agda is the possibility of providing \emph{module parameters}. Importing the rewrite module requires a \emph{term strategy} database:

\Agda{RW}{module-decl}

Not only this makes the code very modular, but also allows one to use different strategies for the same relations, if the need arise.

Finally we have been through all the ingredients we need to build our automatic rewrite tactic. It comes in two flavors. The first one
returns a bunch of information useful for debugging whereas the
later is intended to be used for production.

\Agda{RW}{by'-def}\\
\Agda{RW}{by-def}

\begin{TODO}
  \item wrap it up...
\end{TODO}


