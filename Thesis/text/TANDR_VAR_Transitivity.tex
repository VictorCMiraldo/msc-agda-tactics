Given a transitive rewrite relation, we can, at least, specify how one would
compute the intermediate goals. The implementation, however, is somewhat complex. 
The biggest problem, which is the size of the search space,
does not have an easy workaround. Nevertheless, some contexts have an acceptable run time. 

Lets imagine we want to prove an exchange law for natural numbers and sum, something
in the line of:

\Agda{Transitivity}{exch-law-type}

Since that sum is commutative and associative, this should be pretty straight forward
given such lemmas.

\Agda{Transitivity}{comm-def}\\
\Agda{Transitivity}{id-def}

One possible implementation for \F{exch} could be:

\Agda{Transitivity}{exch-law-impl-1}

Which is a good place to start. We are looking forward to derive something close the above
right-hand-side term. The two arguments for \F{trans} can already be derived using the \F{by}
tactic. Unfortunately, we need a concrete goal to call \F{by}, otherwise we can not infer the
context (abstraction) where the rewrite happens.

The term intersection we defined in the previous chapter is of no use here, since in general,
rewrites do \emph{not} happen in independent subterms. One option is to simply substitute, according
to \F{+-comm}'s type, \emph{one} occurrence of $a + b$ for $b + a$ in $g_1 = x + y$, repeating
the procedure for \F{+-id}. The possibilities are displayed in figure \ref{fig:trans}

\begin{wrapfigure}{r}{0.50\textwidth}
\begin{displaymath}
\scalebox{0.8}{%
\xymatrix{%
  & & (y + 0) + x \\
x + y \ar[r]^{comm} & y + x \ar[ur]^{id} \ar[r]^{id} \ar[dr]^{id} & y + (x + 0) \\
& & (y + x) + 0
}}
\end{displaymath}
\caption{Substitution outcomes}
\label{fig:trans}
\end{wrapfigure}

Note that if we apply first \F{+-id}, the amount of possibilities increases drastically.

The implementation uses the List monad to model non-deterministic computations. The core idea
is modeled by the following three functions. For the more curious reader, we reiterate that
the complete code is available online.

As we already mentioned, the ability to non-deterministically perform a \emph{single} substitution
in a term is central. Given a term $g$ and a pair of terms $m , n$, \F{$g[m , n]$} will return
a term with \emph{one} occurrence of $m$ substituted by $n$. 

\Agda{Transitivity}{subst-def}

Note how the type of \F{$\_[\_] $} requires all terms involved to have the same parameter $A$,
also remember that in the previous section we modeled actions with $n$ explicit parameters as \F{$FinTerm n$}.
Yet, our goal have type \F{$RTerm\; \bot$}. This type mismatch captures the instantiation problem,
depicted in figure \ref{fig:trans}. In fact, we need to chose parameters from our goal to feed
the given action, and only then substitute. 

\Agda{Transitivity}{apply-def}

The function \F{apply g t}, for $\F{t : FinTerm\; n}$, is computed by induction in $n$.
When $n = 0$ it means our action have no more free variables and we can substitute \F{$g[t_1, t_2]$}. 
When $n = k + 1$, however, \F{apply} will non-deterministically
instantiate the last parameter, transforming $t$ to $\F{t' : FinTerm\; k}$, and call itself recursively.
The finite type workaround does not fully work to prove termination, which have to still be explicitly mentioned here.

We finish off with a function that will select, from all the chains produced with \F{$\_[\_]$} and \F{apply},
those that start and finish with the correct terms.

\Agda{Transitivity}{divideGoal-def}

After dividing our goal into a list of intermediate terms, it is just a matter of applying the
tactic to every single one of them and combining everything with a suitable transitivity function.
It is important to note that there are two different explosions of our search space here. The
number of actions we want to apply in a row will determine how \emph{tall} our tree will grow,
whereas a bigger number of parameters of the actions itself will result in a \emph{wider} tree, that is,
more bifurcations at our nodes.

Our approach here is very naive. The \F{apply} function will search every possible instantiation
of a given action's parameters. We could rule out those that do not typecheck, for instance.
This also motivates an addition to Agda's reflection API. A function with type $\F{AgTerm} \rightarrow \F{AgTerm} \rightarrow \F{Bool}$ that returns true when its first argument has its second argument as a possible type.
Such optimization would reduce the search space, unfortunately not in a significant manner.
Using heuristics could also be an interesting optimization.

