\begin{TODO}
  \item intro on rewriting... the diagram here could be cool.
  \item mention eq-reasoning (as a customization of Agda to our needs)
  \item glue to the rest
\end{TODO}


Agda introduced a reflection API in version 2.2.8. Although not a new feature
in the world of functional programming (Lisp's \emph{quoting} and \emph{unquoting} 
and Template Haskell, for instance, are similar techniques) 
it is proving to be very usefull for implementing interesting things. 
One application for reflection, which we chose to explore, is 
the possibility to write custom tactics for proving propositions. 
Somewhat close to how Coq proves propositions.

The representation of an Agda term has type \D{Term}, and can be obtained
by \emph{quoting} it.
The keywords that bridge the world of programs and their abstract representations 
are \K{quoteTerm} and \K{unquote}, inverses of each other. 
The former transforms a normal term into it's \D{Term} representation whereas the 
later does exactly the oposite. Due to Agda's complexity, however, the type \D{Term} is difficult
to handle. A much simpler option is to use a intermediate representation, which we called \D{RTerm},
for implementing the required operations.

After a simpler representation was figured out, we continued by implementing several
\D{RTerm} operations. From these operations we could provide an interesting tactic,
named \F{by}. Whenever the user calls (\K{tactic} (\F{by} action)), we will get information
about the goal and action at that point and try to derive the solution, as long as there exists
a stratey that can handle the goal action pair.

This chapter starts by giving a brief introduction to Agda's Equational Reasoning framework, in section \ref{sec:tandr:reasoning}. Following with Agda's \D{Term} datatype and some of the reflection API, in section \ref{sec:tandr:reflection}. 
On the sequence, section \ref{sec:tandr:representingterms}, we give a description of our \D{RTerm} representation
and explain how we perform the conversion. A presentation of the operations required for 
automatic congruence and substitution guessing follows in section \ref{sec:tandr:termoperations}.
We finish this chapter in section \ref{sec:tandr:rwtactic} where we explain the library entry-point function: the RW function.
Some guidance for when one wants to use our rewrite feature in different domains is also given.
