\documentclass{report}
\usepackage[english]{babel}
\usepackage[utf8x]{inputenc}
\usepackage[T1]{fontenc}
\usepackage{graphicx}

\usepackage{url}
\usepackage{alltt}
\usepackage{listings}
\usepackage{fancyvrb}
\usepackage{float}
\usepackage[usenames,dvipsnames]{color}
\usepackage{enumerate}
\usepackage{amsmath}
\usepackage{amsthm}
\usepackage{amsfonts}
\usepackage{amssymb}
\usepackage{makeidx}
\usepackage{parskip}
\usepackage{multirow}
\usepackage{moreverb}
\usepackage{proof}
\usepackage[all]{xy}
\usepackage{catchfilebetweentags}

\usepackage{agda} 

%\setmainfont{XITS}
%\setmathfont{XITS Math}

\long\def\red#1{\color{red} #1 \color{black}}

\newenvironment{TODO}{%
  \color{blue} \itshape \begin{itemize}
}{%
  \end{itemize}
}

\newcommand{\warnme}[1]{{\color{red} \textbf{$[$} #1 \textbf{$]$}}}

% empty env, maybe later we can add some style to it.
\newenvironment{agdacode}{%
\vspace{.5em}
\hspace{1em}
\begin{minipage}[t]{.8\textwidth}
}{%
\end{minipage}
\vspace{.5em}
}

\newcommand{\Agda}[2]{%
\begin{agdacode}
\ExecuteMetaData[code/agda/excerpts/#1.tex]{#2}
\end{agdacode}
}


\newenvironment{agdacodeInd}{%
\vspace{.5em}
\hspace{-1em}
\begin{minipage}[t]{.8\textwidth}
}{%
\end{minipage}
\vspace{.5em}
}

\long\def\ignore#1{}

%% Notation
\newcommand{\inlagda}[1]{$\mathsf{\small #1}$}
\newcommand{\catname}[1]{\textbf{#1}}

%%%% Relational Notation
\newcommand{\releq}{\equiv_r}

%% Agda shortcuts
\newcommand{\D}[1]{\AgdaDatatype{#1}}
\newcommand{\F}[1]{\AgdaFunction{#1}}
\newcommand{\K}[1]{\AgdaKeyword{#1}}
\newcommand{\N}[1]{\AgdaSymbol{#1}}
\newcommand{\IC}[1]{\AgdaInductiveConstructor{#1}}
\newcommand{\ICArgs}[2]{\AgdaInductiveConstructor{#1}$\; #2 $}
\newcommand{\DArgs}[2]{\D{#1}$\; #2 $}

%% Operations
\def\conc{++}

%for some reasons, LaTeX does not like unicode chars outside math mode.
%therefore we have to define them here.
\newcommand{\textrho}{$\rho$}
\newcommand{\textLambda}{$\Lambda$}
\newcommand{\textpi}{$\pi$}

% And some others that actually require the unicode declaration
\DeclareUnicodeCharacter {10627}{\{\hspace {-.2em}[}
\DeclareUnicodeCharacter {10628}{]\hspace {-.2em}\}}
\DeclareUnicodeCharacter {8759}{::}
\DeclareUnicodeCharacter {8718}{$\square$}

\title{Proof by Rewritting in Agda\\ \small{Pre Dissertation Report}}
\author{Victor Cacciari Miraldo\\[2cm] %
\small{supervised by} \\
Prof. José Nuno Oliveira and Prof. Wouter Swierstra \\
University of Minho and Utrecht University}


\begin{document}
\maketitle
\tableofcontents

\theoremstyle{plain}
\newtheorem{thm}{Theorem}[chapter]
\newtheorem{crl}{Corolary}[chapter]
\newtheorem{prob}{Problem}[chapter]
\newtheorem{prop}{Proposition}[chapter]

\theoremstyle{definition}
\newtheorem{lemma}{Lemma}[chapter]
\newtheorem{mydef}{Definition}[chapter]
\newtheorem{notation}{Notation}[chapter]

\theoremstyle{remark}
\newtheorem{nota}{Note}[chapter]

\chapter{Prelude}
\label{chap:prelude}

  \section{Introduction}
  \label{sec:prelude:introduction}
  Although formal logic can be traced back to Aristotle, most of the
groundbreaking work was done around the end of the 19th and early 20th centuries:
the specification of propositional calculus; what we now know as predicate logic and other developments. The quest for bootstrapping mathematics, that is, formalizing mathematics in formal logic, was
being pursued by many. A notorious attempt was made by Russel and Whitehead in \emph{Principia Mathematica}, \cite{Whitehead1912}, where they believed that all mathematical truths
could be derived from inference rules and axioms, therefore opening up the question of automated reasoning. Yet, in 1931, Kurt G\"{o}del published his famous first and second incompleteness theorems. In a (very small) nutshell, they state that there are some truths that are not provable, regardless of the axiomatic system chosen. This question was further addressed by Alonzo Church and Alan Turing, in the late 1930s. This is when a definite notion of computability first arose (in fact they gave two, independent, definitions).

Armed with a formal notion of computation, mathematicians could finally start to explore this newly
founded world, which we call Computing Science nowadays. Problems started to be categorized in 
different classes due to their complexity. In fact, some problems could now be proven to be \emph{unsolvable}.
Whenever we encounter such problem, we must work with approximations and subproblems that we
know we can compute a solution for. 

Given a formula in a logic system, the question of whether or not such a formula is true can vary from trivial to impossible. The simplest case is, of course, propositional logic, where validity is decidable but not at all that interesting
for software verification in general. We need more expressive formal systems in order to encode software specifications, as they usually involve quantification or even modal aspects.  A \emph{holy grail} for formal verification would be the construction of a fully automatic theorem prover, which is very hard (if not impossible) to achieve. Instead, whenever the task requires an expressive system, we could only provide a guiding hand to our fellow mathematicians. This is what we call a \emph{Proof Assistant}.

Proof Assistants are highly dependent on which logic they can understand. With the development of more
expressive logics, comes the development of better proof assistants. Such tools are used to mitigate
simple mistakes, automate trivial operations, and make sure the mathematician is not working
on any incongruence. Besides the obvious verification of proofs, a proof assistant also opens up
a lot of room for proof automation. By automating mechanical tasks in the development of 
critical software or models we can help the programmer to focus on what is really important
rather than making he write boilerplate code that is mechanical in nature.

As we said above, proof assistants depend on the logic they run on top of. Some of them,
however, support some form of metaprogramming. That is, we can write programs that generate
programs. This is achieved by having the programming language itself as a first class datatype.
Examples include LISP and Prolog. One of the goals of this project is to explore how can
we exploit such feature to automate the repetitive task of rewriting terms for other,
proven to be equal, terms.

The tool of choice for this project is the Agda language, developed at Chalmers \cite{norell07}. Agda uses a intensional
variant of Martin-L\"{o}f's theory of types and provides a nice interactive construction feature. After
the Curry-Howard isomorphism \cite{Howard01}, interactive program construction and assisted theorem proving are essentially
the same thing. The usual routine of an Agda programmer is to write some code, with some \emph{holes} for
the unfinished parts, and then ask the typechecker which types should such \emph{holes} have. This is done
interactively. Yet, trivial operations to write on paper usually require additional code
for discharging in Agda. One such example is its failure to automatically recognize $i + 1$ and
$1 + i$ as returning the same value. The usual strategy is to rewrite this subterm of our goal
using a commutativity proof for $+$. 

Small rewrites are quite simple to perform using the \K{rewrite} keyword. If we need to perform
equational reasoning over complex formulas, though, we are going to need to specify the
substitutions manually in order to apply a theorem to a subterm. The main objective of this project is
to work around this limitation and provide a smarter rewriting mechanism for Agda. Our main 
case study is the equational proofs for relational algebra \cite{Bird97}, which also involves
the construction of a relational algebra library suited for rewriting.

  
  \section{The Agda language}
  \label{sec:prelude:agdalanguage}
  Although functional languages have been receiving great attention
and a fast evolution in recent years, one of the biggest jumps
has been the introduction of dependent types. Agda\cite{norell07} is one such language, other 
big contributions being \emph{Epigram} \cite{mcbride05} and \emph{Coq} \cite{bertot06}. 

In languages like Haskell or ML, where a Hindley-Milner based algorithm is used
for type checking, values and types are clearly separated. Within dependent
types, though, the story changes. A type can depend on any value the programmer
wishes for. A classical example are the fixed size vectors \inlagda{Vec\; A\; n}, where \inlagda{A}
is the type of the elements and \inlagda{n} is the length of the vector. Readers familiar
with C may argue that C also has fixed size arrays. The difference is that
this \inlagda{n} need not to be known at compile time, it can be an arbitrary term (as long
as it has type \inlagda{Nat}). That is, there is no difference between types and values,
they are all sets, explained in the sequel).

In this chapter introduces the basics of Agda and show some examples in both
the programming and the proof theoretic sides of the language. Later on, in 
section \ref{sec:background:martinlof}, we will see the theory in which Agda is built 
on top of, and whereby a number of concepts introduced below will become clearer. 
Some background on the $\lambda$-calculus and the Curry-Howard isomorphism is
presented in section \ref{sec:background:lambdacalculus}.

\subsection{Peano Naturals}

In terms of programing, Agda and Haskell are very close. Agda's syntax was
inspired by Haskell and, being also a functional language, a lot of
the programming techniques we need to use in Haskell also apply for an Agda program.
Knowledge of Haskell is not strictly necessary, but of a great value for a better
understating of Agda.\\

The \emph{Hello World} program in Agda is to encode Peano's Natural numbers:

\Agda{Basic}{NAT}

The data declarations in Agda are very similar to a GADTs\cite{Xi2003} in Haskell. Remember that in Agda types and values are the same thing.
The data declaration above states that \inlagda{Nat} is \emph{of type} 
\inlagda{Set}. This \inlagda{Set} is the type of types, to be defined later. 
For now, think of it as Haskell's $*$ kind.


As expected, definitions are made by structural induction over the
data type. Pattern matching is the mechanism of choice here.

\Agda{Basic}{PLUS-MUL}

There are a few points of interest in the above definitions. The underscore pattern
(\_) has the same meaning as in Haskell, it matches anything. The underscore in
the symbol name, though, indicates where the parameters should be relative to the symbol name. 
Agda supports mixfix operators and has full UTF8 support. It is possible to
apply a mixfix operator in normal infix form: \inlagda{a + b = \_+\_\; a\; b}.

\subsection{Propositional Logic, a fragment}

Let us continue our exposition of Agda by encoding some propositional logic. 
We first need two sets for representing the truth values:

\Agda{Basic}{TOP-BOT}

The truth proposition has only one proof, and this proof is trivial. So we define $\top$ as a set
with a single, constant, constructor. The absurd is modeled as a set with no constructors, therefore no elements. In the theory of types,
this means that there is no proof for the proposition $\bot$, as expected. Note the definition
of \N{$\bot-elim$}, its type is, for all sets $A$\footnote{
%%%%% BEGIN FOOTNOTE
Implicit parameters are enclosed in brackets, this can be read as universal quantification in
the type level.
%%%%% END FOOTNOTE
}, $\bot \rightarrow A$. Which exactly captures
the notion of proof by contradiction in logic. The definition is trickier, though.
Agda's empty pattern, \inlagda{()}, is used to discharge a contradiction. It tells
that there is no possible pattern in such equation, and we're discharged of writing
a right-hand side.

%The top set, $\top$, on the other
%hand, is written as a set with only one constructor, proving $\top$ is trivial. Yet, for reasons that will be clear 
%later, we chose to encode $\top$ as a record with no fields, and therefore only one constructor, \inlagda{record\{\}} in Agda's syntax.
%Following Agda's standard library notation, let's call it \D{Unit}.

%\Agda{Basic}{UNIT}

Let us now show how to encode a fragment of propositional logic in this framework. Let's take
conjunction as a first candidate. We begin by declaring the set that represents conjunctions:

\Agda{Basic}{CONJUNCTION}

So, \D{A /\symbol{92} B} contains elements with a proof of $A$ and a proof of $B$. It's elements
can only be constructed with the \IC{<\_,\_>} constructor. The reader familiar with Natural Deduction
might recognize the following trivial properties. Given that $D$ is a set:

\begin{enumerate}[i)]
  \item \emph{Formation} rules for $D$ express the conditions under which $D$ is a set.
        In our example, whenever $A$ and $B$ are sets, then so is \D{A /\symbol{92} B}.
  \item \emph{Introduction} rules for $D$ define the set $D$, that is, they specify how the
        canonical elements of $D$ are constructed. For the conjunction case, 
        the elements of \D{A /\symbol{92} B} have the form \IC{< a , b >}, where $a \in A$
        and $b \in B$. 
  \item \emph{Elimination} rules show how to prove a proposition about an arbitrary element in $D$.
        They're very closely related to structural induction. The Agda equivalent is pattern matching,
        where we deconstruct, or eliminate, an element until we find the first constructor. Note that
        the proofs of \F{/\symbol{92}-elim} are simply pattern matching.
  \item \emph{Equality} rules gives us the equalities which are associated with $D$. Without diving
        too much, in our simple case, \inlagda{< a , b > == < a' , b' >} whenever \inlagda{a == a'}
        and \inlagda{b == b'}.
\end{enumerate}

As we shall see in section \ref{sec:background:martinlof}, these properties come for free whenever we introduce 
a new set forming operation, or, a data declaration in Agda. In fact, for each set forming operation $D$
we have those four kinds or rules.

An analogous technique can be employed to encode disjunction. Note the similarity with
Haskell's \D{Either} datatype (they are the same, in fact). The elimination rules for
disjunction is a proof by cases. 

\Agda{Basic}{DISJUNCTION}

As a trivial example, we can prove that conjunction distributes
over disjunction.

\Agda{Basic}{DISTR}

Using our fragment of natural deduction, we can start proving things. Let's consider a simple
proposition saying that an element of a nonempty list $l_1 \conc l_2$, where $\conc$ is concatenation, is
in $l_1$ or in $l_2$. For this all that is required is: to encode lists, provide a definition for $\conc$,
encode a \emph{view} of the elements in a list and finally prove our proposition.

Defining the type of lists is a boring task, and the definition is just like
its Haskell counterpart. Let us just import from the standard library and go to the
interesting part right away.

\Agda{Basic}{IN-LIST}

Looking closely to this code: we have a set forming operation \D{In}, that receives
one implicit parameter $A$ and is indexed by an element of $A$ and a $List\ A$. Indeed, \DArgs{In}{x\; l}
is the proposition that states that $x$ is an element of $l$. How do we construct such a proof?
There are two ways of doing so! Either $y$ is in $l$'s head, as captured by the \IC{InHead}
constructor, or it is in $l$'s tail, and for that we require a proof of such statement, in the \IC{InTail}
constructor. The reader may wonder about the statement \DArgs{In}{y\; []}, which should be impossible. One can see,
just from the types, that we cannot construct such a statement. The result of both of \IC{In} constructors
involve non-empty list and they are the only constructors we are allowed to use. So, the \emph{non-empty list}
requirement in the proposition we're trying to prove comes for free.

We proceed to showing how to state the proposition mentioned above and carry out its proof step by step, as this is an interesting example.
The proposition is stated as follows,

\Agda{Basic}{inDistr-decl}

The variables enclosed in normal parentheses are called \emph{telescopes}, and they introduce
the dependent function type $(x : A) \rightarrow B\;x$, that is, $x$ can appear on the right-hand side of
the function type constructor, $\rightarrow$. Therefore, we need an implicit set $A$, two lists of $A$,
an element of $A$ and a proof that such element is in the concatenation of the two lists.
The proof follows by \emph{induction} on the first list. 

\Agda{Basic}{inDistr-1}

If the first list is empty, $[] \conc l_2$ reduces to $l_2$. So, prf has type \DArgs{In}{x\; l_2}, which
is just what we need to create a disjunction.

\Agda{Basic}{inDistr-2}

If it's not empty, though, then it has a head and a tail. Now we have to also pattern-match
on $prf$. If it says that our element is in the head of our list, the result is also very simple.
The repetition of the symbol $x$ in the left hand side is allowed as long as all duplicate occurrences
are preceded by a dot. They're called dotted patterns and tell Agda that this is \emph{the only possible}
value for that pattern. We can see that by reasoning with \IC{InHead} type, \DArgs{In}{x\; (x :: l)}. 
So, if $l_1 = (x :: l)$ and we have $prf = In\; x\; (x :: l)$, that is, a proof that $x$ is in $l_1$'s head,
we can only be looking for $x$. 

\Agda{Basic}{inDistr-3}

In case our element is not in $l_1$'s head, it might be either in the rest of $l_1$ or in $l_2$.
Note that we pass a \emph{structurally smaller} proof to the recursive call. This is a condition required by
Agda's termination checker.

\subsection{Closing Remarks}

In this small introduction to Agda we could see a tiny bit of what the language is capable of,
both in its programming side and its proof assistant side. All the concepts were introduced
informally here, with the intention of not overwhelming a unfamiliar reader. A lot of resources
are available in the Internet at the Agda Wiki page \cite{AgdaTutorials}. 




  
  \section{The Problem}
  \label{sec:prelude:theproblem}
  As we could see in section \ref{sec:prelude:agdalanguage}, Agda is a very expressive language
and it allows us to build smaller proofs than the great majority of proof assistants available.
The mixfix feature gives the language a very customizable feel, one application is the
equational reasoning framework. In the following illustration we prove the associativity of
the concatenation operation.

\Agda{Basic}{++-assocH}

The notation is very clear and understandable, it indeed looks very much
to what a \emph{squiggolist}\footnote{
%%%% BEGIN FOOTNOTE
\emph{Squiggol} is a slang name for the Bird–Meertens formalism, due to the squiggly symbols it
uses.
%%%% END FOOTNOTE
} would write on paper. One of the main downsides to it, which is also inherent to Agda in general,
is the need to specify every single detail of the demonstration, even the trivial ones.

  
\chapter{Background}
\label{chap:background}
In this chapter we'll introduce some background notions. The ideia is to keep this thesis
mathematically self contained, but for a full understanding of this kind of topics, further
reading is necessary. The reader familiar with $\lambda$-calculus and the Curry-Howard isomorphism for simply typed terms can skip this chapter.

As we already mentioned in section \ref{sec:prelude:agdalanguage} Agda is both a programming language and a proof assistant. The programming side of Agda is a pure functional languague and is built on top of the (dependently typed) $\lambda$-calculus, such formalism will be briefly presented in section \ref{sec:background:lambdacalculus}. The proof assistant view, on the other hand, lies on top of the Curry-Howard isomorphism. We'll introduce this topic in section \ref{sec:background:curryhoward}. 


  \section{Notes on $\lambda$-calculus and Types}
  \label{sec:background:lambdacalculus}
  What we call $\lambda$-calculus is a collection of various formal systems based 
in the notation invented by Alonzo Church in \cite{Church01,Church02}. 
Church solved the famous \textit{Entscheidungsproblem} (from German, \textit{decision problem}) 
proposed by David Hilbert, in 1928. The challenge consisted in providing an algorithm capable
of determining whether or not a given mathematical fact was valid in a given language. 
Church proved that there is no solution for such problem, that is, it's an undecidable problem.

One of Church's main objectives was to build a formal system for the foundations of
Mathematics, just like Martin-Löf's type theory, which was be presented later, around 1970. 
Church dropped his work when his basis was found to be inconsistent. Later on it was found
that there were ways of making it consistent again, with the help of types.

The notion of type is paramount for this thesis as a whole. This notion arises when we want 
to combine different terms in a given language, for instance, it makes no sense to try
to compute $\int \mathbb{N}\;\mathrm{d}x$. Although syntatically correct, it's subterms have
different \emph{types} and, therefore, are not compatible. A type can be seen as a
categorization of terms.

For a torough introduction of the lambda-calculus, the reader is directed to \cite{Barendregt01,Hindley01}.
The goal of this chapter is to make it easier to the reader to see how Martin-Löf's type theory
was born. There are a lot of connections with it's cousin the $\lambda$-calculus and seeing this
connections might give a better understanting. 

\subsection{The $\lambda$-calculus}

\begin{mydef}[Lambda-terms] Let $\mathcal{V} = \{v_1, v_2, \cdots\}$ be a infinite set of
variables, $\mathcal{C}=\{c_1, c_2, \cdots\}$ a set of constants such that 
$\mathcal{V} \cap \mathcal{C} = \emptyset$. The set $\Lambda$ of lambda-terms is 
inductively defined by:
\begin{description}
  \item[Atoms]
        \[ \mathcal{V} \cup \mathcal{C} \subset \Lambda \]
  \item[Application]  
        \[ \forall M, N \in \Lambda.\; (M N) \in \Lambda \]
  \item[Abstraction]
        \[ \forall M \in \Lambda, x \in \mathcal{V}.\; (\lambda x . M) \in \Lambda \]
\end{description}
\end{mydef}

Let's make some conventions that will be usefull throughout this document. We'll usually denote
terms by uppercase letters $M, N, O, P, \cdots$ and variables by lowercases $x, y, z, \cdots$.
Application is left associative, that is, the term $M N O$ represents $((M N) O)$ whereas
abstractions are right associative, so, $\lambda x y. K$ represents $(\lambda x . (\lambda y.K))$.

Suppose we have a term $\lambda yz.x(yz)$, we say that variables $y$ and $z$ are bounded variables
and $x$ is a free variable (there is no abtraction binding it visible). For now on, we'll 
use Barendregt's convenssion for variables and assume that terms do not have any name clashing.
In fact, whenever we have two terms $M$ and $N$ that only differ in the naming of their variables,
for instance $\lambda x . x$ and $\lambda y . y$, we say that they are $\alpha$-convertible.\\

\begin{mydef}[Substitution] Let $M$ and $N$ be lambda-terms where $x$ has a free occurence
in $M$. The substitution of $x$ by $N$ in $M$, denoted by $[N/x]M$ is, informally, the result
of replacing every $x$ in $M$ by $N$. It is defined by induction on $M$ by:
\begin{eqnarray*}
   {[N/x]} x & = & N \\
   {[N/x]} y & = & y, \; y \neq x \\
   {[N/x]}(M_1 M_2) & = & {[N/x]M_1\; [N/x]M_2} \\
   {[N/x]}(\lambda y . M) & = & (\lambda y . {[N/x]M})
\end{eqnarray*}
\end{mydef}

\subsection{Beta reduction and Confluency}

We are now equiped with both a notion of term and a formal definition of substitution, we can
model the notion of computation. The intuitive meaning is very simple. Imagine a normal
function $f(x) = x + 3$ and suppose we want to compute $f(2)$. All we have to do is to
substitute $x$ for $2$ in the body of $f(x)$, resulting in $2+3$.

This notion is followed to the letter in the $\lambda$-calculus. A term with the form $(\lambda x . M)N$ is
called a $\beta$-redex and can be reduced to $[N/x]M$. If a given term has no $\beta$-redexes we say
it is in $\beta$-normal form.

\newcommand{\betaright}{\rightarrow_{\beta}}
\newcommand{\betarightright}{\twoheadrightarrow_{\beta}}
\begin{mydef}[$\beta$-reduction]
Let $M, M'$ and $N$ be lambda-terms and $x$ a variable. Let $\betaright$ be the following binary relation
over $\Lambda$ defined by induction on $M$.
\begin{center}
    \begin{tabular}{c c}
      \infer{(\lambda x . M) N \betaright [N/x]M}{} & 
      \infer{(\lambda x.M) \betaright (\lambda x. M')}{M \betaright M'} \\[1cm]
      \infer{MN \betaright M'N}{M \betaright M'} & 
      \infer{NM \betaright NM'}{M \betaright M'} \\[1cm]
    \end{tabular}
  \end{center}
\end{mydef}

We denote by $M \betarightright N$ when $N$ is obtained through zero or more $\beta$-reductions 
from $M$.\\

\begin{mydef}[$\beta$-equality]
Let $M$ and $N$ be lambda terms, we say that $M$ and $N$ are $\beta$-equal, and denote by 
$M =_{\beta} N$ if $M \betarightright N$ or $N \betarightright M$.\\
\end{mydef}

\begin{thm}[Confluency]
Let $M, N_1$ and $N_2$ be lambda terms. If $M \betaright N_1$ and $M \betaright N_2$ then there
exists a term $Z$ such that $N_i \betarightright Z$, for $i \in \{1, 2\}$.\\
\end{thm}

\begin{thm}[Church-Rosser]. Let $M$ and $N$ be lambda-terms such that $M =_{\beta} N$. Then there
exists a term $Z$ such that $M \betarightright Z$ and $N \betarightright Z$.
\end{thm}

Note that the aforementioned results are of enormous relevance not only for the $\lambda$-calculus,
but for similar formalisms too. They allow us to prove that, for instance, the normal form
of a lambda-term (if it exists\footnote{%
%%%%% BEGIN FOOTNOTE
There are terms that do not have a normal form. A classical example is $(\lambda x . xx)(\lambda x . xx)$.
The reader is invited to compute a few $\beta$-reductions on it.
%%%%% END FOOTNOTE
}) is unique. In fact, lambda-calculus consistency is proved using these results \cite{Barendregt01}.

\subsection{Simply typed $\lambda$-calculus}

\begin{mydef}[Type]
Let $\mathcal{C}_{\mathcal{T}} = \{ \sigma, \sigma', \cdots \}$ be a set of atomic types,
we define the set $\mathbb{T}$ of simple types by indution:
\begin{enumerate}[i)]
  \item $\mathcal{C}_{\mathcal{T}} \subset \mathbb{T}$
  \item $\forall \sigma, \tau \in \mathbb{T}. \; (\sigma \rightarrow \tau) \in \mathbb{T}$.
\end{enumerate}
\end{mydef}

In a programming context, we're surrounded by variable declarations. Some languages (the strongly-typed ones)
expect some information about the type of such variables. This is what we call a context.
Formally, a context is a set $\Gamma \subseteq \mathcal{V} \times \mathbb{T}$, whose elements
are denoted by $(x : \sigma)$.

This allows us to define the notion of derivation and derivability, and we're almost closing the
gap between programming and logic.\\

\begin{mydef}[Derivation]
\label{def:simpletypederivation}
We define the set of all type derivations by induction in the target lambda-term:
\begin{enumerate}[i)]
  \item
    \[\vcenter{\infer[(Ax)]{\Gamma \vdash x:\sigma}{}} \]
    
  \item 
    \[\vcenter{\infer[(I\rightarrow)]{\Gamma \vdash (\lambda x.M) : (\tau \rightarrow \sigma)}
							{\infer*{\Gamma, x:\tau \vdash M:\sigma}{}}} \]

  \item 
    \[\vcenter{
			\infer[(E\rightarrow)]
				{\Gamma \vdash MN : \sigma}
				{
					\infer*
						{\Gamma \vdash M:(\tau \rightarrow \sigma)}
						{}
				&
					\infer*
						{\Gamma \vdash N:\tau}
						{}
				}
		
		}\]
\end{enumerate}
\end{mydef}

\begin{mydef}[Derivability]
Let $\Gamma$ be a context, $M$ a lambda-term and $\sigma$ a type. We say that
the sequent $\Gamma \vdash M : \sigma$ is derivable is there exsits a derivation with
such sequent as it's conclusion.
\end{mydef}

The simply-typed $\lambda$-calculus is a model of computation. It has the same expressive power as
the Turing Machine for expressing computability notions. This is a very well studied subject
and the references provided in this chapter are a compilation of everything that
has been studied so far. For more typed variations of the $\lambda$-calculus we refer the reader
to \cite{Barendregt03}. We're not interested in that aspect of the $\lambda$-calculus, 
though. We want to explore it's connection with Mathematical logic. In the next subsection
we'll explore this connection.

\subsection{The Curry-Howard Isomorphism}

On one hand we have the models of computation, on the other hand we have the proof systems.
A a first glance, they look like very different formalisms, but they turned out to be
structurally the same. Let $M$ be a term and $\Gamma$ a context such that
$\Gamma \vdash M : \sigma$ is derivable. We can look at $\sigma$ as a propositional formula\footnote{
%%%% BEGIN FOOTNOTE
Remeber that the implication, here denoted by $\subset$, forms a minimal complete connective set
and is, therefore, enough to express the whole propositional logic.
%%%% END FOOTNOTE
} and to $M$ as a proof of such formula. There are other ways to show this connection,
but I'll illustrate it using the Natural Deduction\cite{Prawitz01} (we'll denote the
propositional implication by $\supset$). Let's put the rules presented in definition
\ref{def:simpletypederivation} side-by-side with the Axiom, $\supset$-elimination and
$\supset$-introduction rules from Natural Deduction;

\begin{center}
\begin{align*}
	\text{Natural Deduction} & \hspace{3cm} \text{Type Derivation} \\
	\sigma \hspace{1.4cm} & \hspace{3cm} \infer[(Ax)]{\Gamma \vdash M : \sigma}{}  \\[0.5cm]
	\infer[(I\supset)]{\tau \supset \sigma}{\infer*{\sigma}{[\tau]}}
		& \hspace{3cm} 
		\infer[(I\rightarrow)]{\Gamma \vdash (\lambda x . M) : (\tau \rightarrow \sigma)}
							  {\Gamma, x:\tau \vdash M : \sigma} \\[0.5cm]					  
	\infer[(E\supset)]{\sigma}{\tau \supset \sigma & \tau}
		& \hspace{3cm}
		\infer[(E\rightarrow)]{\Gamma \vdash MN : \sigma}
			{ \Gamma \vdash M : (\tau \rightarrow \sigma) & \Gamma \vdash N : \sigma}
\end{align*} 
\end{center}

This semingly shallow equivalence is a remarkable result in Computing Science, discovered
by Curry and Howard in \cite{Curry01,Howard01}. This was the starting point for the
first proof checkers, since checking a proof is the same as typing a lambda-term. If the
term is typeable, then the proof is valid. Up to this point we only presented the simpler
version of this connection. We'll later on build on top of it and add all ingredients
for working over first-order logic and using Agda as our proof assistant. Under the curtains,
all Agda does is type-checking terms.

The understanding of this connection is of big importance for writing proofs and programs in
Agda (or any other proof-assistant based on the Curry-Howard isomorphism, for that mater).



  
  \section{Martin-Löf's Type Theory}
  \label{sec:background:martinlof}
  Type theory was originally developed with the goal of offering a clarification, or basis,
for constructive Mathematics. However, unlike most other formalizations of mathematics, it is
not based on first order logic. Therefore, we need to introduce the symbols and
rules we'll use before presenting the theory itself. The heart of this interpretation of
proofs as programs is the Curry-Howard isomorphism, already explained in section \ref{sec:background:lambdacalculus}.

Martin-L\"{o}f's theory of types \cite{lof84} is an extension of regular type theory. This extended
interpretation includes universal and existential quantification. 
A proposition is interpreted as a set whose elements
are proofs of such proposition. Therefore, any true proposition is a non-empty set and any false proposition
is a empty set, meaning that there is no proof for such proposition. Apart from \emph{sets as propositions},
we can look at sets from a \emph{specification} angle, and this is the most interesting view for programming.
A given element $a$ of a set $A$ can be viewed as: a proof for proposition $A$; a program satisfying the
specification $A$; or even a solution to problem $A$.

This chapter we'll explain the basics of the theory of types (in its \emph{intensional} variation)
trying to establish connections with the Agda language. It begins by providing some basic notions
and the interpretation of propositional logic into set theory. We'll follow with the notion of arity,
which differs from the canonical meaning, finishing with a small discussion on the dependent product
and sums operators, which closes the gap to first order logic. The interested reader should continue
with \cite{nords90} or, for a more practical view, \cite{wouter08,bove2009}

\subsection{Constructive Mathematics}
\label{subsec:martinlof:constructivemathematics}

The line between Computer Science and Constructive Mathematics is very thin. The primitive object is
the notion of a function from a set $A$ to a set $B$. Such function can be viewed as a program that,
when applied to an element $a \in A$ will construct an element $b \in B$. This means that every
function we use in constructive mathematics is computable. 

Using the constructive mindset to prove things is also very closely related to building a computer program.
That is, to prove a proposition $\forall x_1,x_2 \in A \; . \; \exists y \in B \; . \; Q(x_1, x_2, y)$ for a given
predicate $Q$ is to give a function that when applied to two elements $a_1, a_2$ of $A$ will give an element $b$ in $B$
such that $Q(a_1, a_2, b)$ holds. 

\subsection{Propositions as Sets}
\label{subsec:martinlof:propositionsassets}

In classical mathematics, a proposition is thought of as being either true or false, and it doesn't
matter if we can prove or disprove it. On a different angle, a proposition is constructively true
if we have a \emph{method} for proving it. A classical example is the law of excluded middle, $A \vee \neg A$,
which is trivially true since $A$ can only be true or false. Constructively, though, a method for proving a disjunction
must prove that one of the disjuncts holds. Since we cannot prove an arbitrary proposition $A$, we have
no proof for $A \vee \neg A$. 

Therefore, we have that the constructive explanation of propositions is built in terms of proofs, and
not an independent mathematical object. The interpretation we are going to present here is due
to Heyting at \cite{Heyting71}.

\paragraph{Absurd,} $\bot$, is identified with the empty set, $\emptyset$. That is, a set with no elements
or a proposition with no proof.

\paragraph{Implication,} $A \supset B$ is viewed as the set of functions from $A$ to $B$, denoted $B^A$. That is,
a proof of $A \supset B$ is a function that, given a proof of $A$, returns a proof of $B$.

\newcommand{\pone}{\pi_1}
\newcommand{\ptwo}{\pi_2}
\paragraph{Conjunction,} $A \wedge B$ is identified with the cartesian product $A \times B$. That is, a proof
of $A \wedge B$ is a pair whose first component is a proof of $A$ and second component is a proof of $B$.
Let us denote the first and second projections of a given pair by $\pone$ and $\ptwo$.
The elements of $A \times B$ are of the form $(a, b)$, where $a \in A$ and $b \in B$.

\newcommand{\ione}{i_1}
\newcommand{\itwo}{i_2}
\paragraph{Disjunction,} $A \vee B$ is identified with the disjoint union $A + B$. A proof
of $A \vee B$ is either a proof of $A$ or a proof of $B$. The elements of $A + B$ are of the
form $\ione\; a$ and $\itwo\; b$ with $a \in A$ and $b \in B$.

\paragraph{Negation,} $\neg A$, can be identified relying on its definition
on the minimal logic, $A \supset \bot$

So far, we defined propositional logic using sets (types) that are available in almost every
programming language. Quantifications, though, require operations defined over a 
family of sets, possibly \emph{depending} on a given \emph{value}. The intuitionist explanation
of the existential quantifier is as follows:

\paragraph{Exists,} $\exists a \in A \; . \; P(a)$ consists of a pair whose first
component is one element $i \in A$ and whose second component is a proof of $P(i)$. More generally,
we can identify it with the disjoint union of a family of sets, denoted by $\Sigma(x \in A, B(x))$,
or just $\Sigma(A, B)$. The elements of $\Sigma(A, B)$ are of the form $(a, b)$ where $a \in A$ and
$b \in P(a)$.

\paragraph{For all,} $\forall a \in A \; . \; P(a)$ is a function that gives a proof of $P(a)$
for each $a \in A$ given as input. The correspondent set is the cartesian product of a family
of sets $\Pi(x \in A, B(x))$. The elements of such set are the aforementioned functions. The same
notation simplification takes place here, and we denote it by $\Pi(A, B)$. The elements of such set
are of the form $\lambda x\;.\;b(x)$ where $b(x) \in B(x)$ for $x \in A$. 

\subsection{Expressions}

Thus far now we have identified the sets needed to express first order formulas, but we
did not mention what an expression is. In fact, in the theory of types, an expression is a
very abstract notion. We are going to define the set $\mathcal{E}$ of all expressions by induction shortly.

It is worth remembering that Martin-L\"{o}f's theory of types was intended to be a foundation
for mathematics. It makes sense, therefore, to base our definitions in standard mathematical expressions.
For instance, consider the syntax of an expression in the CCS calculus \cite{Milner80} $E \sim \Sigma \{ a.E' \;\mid\; E \xrightarrow{\;a\;} E' \}$\footnote{%
%%% FOOTNOTE
Expansion Lemma for CCS; States that every process (roughly a LTS) is equivalent to the sum of its derivatives.
The notation $E \xrightarrow{\;a\;} E'$ states a transition from $E$ to $E'$ through label $a$.
%%% END FOOTNOTE
}; we have a large range of syntactical elements that we don't usually think about when writing
mathematics on paper. For instance, the variables $a$ and $E'$ works just as placeholders, which
are \emph{abstractions} created at the predicate level in the common set-by-comprehension syntax. We then
have the \emph{application} of a summation $\Sigma$ to the resulting set, where this symbol represents
a \emph{built-in} operation, just like the congruence $\_\sim\_$. These are exactly the elements
that will allow us to build expressions in the theory of types.\\

\begin{mydef}[Expressions]\hfill
\begin{description}
  \item[Application;]
    Let $k, e_1, \cdots, e_n \in \mathcal{E}$ be expressions of suitable arity. The application of $k$ to $e_1, \cdots, e_n$ denoted by $(k\;e_1\;\cdots\;e_n)$,
    is also an expression.
  
  \item[Abstraction;]
    Let $e \in \mathcal{E}$ be an expression with free occurrences of a variable $x$.
    We denote by $(x)e$ the expression where every free occurrence of $x$ is
    interpreted as a hole, or context. So, $(x)e \in \mathcal{E}$.
  
  \item[Combinations;]
    Expressions can also be formed by combination. This is a less common construct.
    Let $e_1, \cdots, e_n \in \mathcal{E}$, we may form the expression:
    \[
      e_1,e_2,\cdots,e_n
    \]
    which is the combination of the $e_i$, $i \in \{1,\cdots,n\}$. 
    This can be though of tupling things together. Its main use is for
    handling complex objects as part of the theory. Consider the
    scenario where one has a set $A$, an associative closed operation $\oplus$ and
    a distinguished element $a \in A$, neutral for $\oplus$. We could talk
    about the monoid as an expression: $A, \oplus, a$.
  
  \item[Selection;]
    Of course, if we can tuple things together, it makes sense to be able to
    tear things apart too. Let $e\in\mathcal{E}$ be a combination with
    $n$ elements. Then, for all $i \in \{1, \cdots, n\}$, we have the
    selection $e.i \in \mathcal{E}$ of the $i$-th component of $e$.
  
  \item[Built-in's;]
    The built-in expressions are what makes the theory shine. Yet, 
    they change according to the flavor of the theory of types one is handling.
    They typically include $zero$ and $succ$ for Peano's encoding of the Naturals, 
    together with a recursion principle $natrec$; Or $nil$, $cons$ and $listrec$
    for lists; Products and Coproduct constructors are also built-ins: $\langle\_,\_\rangle$
    and $inl, inr$ respectively.
\end{description}
\end{mydef}\hfill

\begin{mydef}[Definitional Equality]
Given two expressions $d, e \in \mathcal{E}$, Should they be syntactical synonyms, we say
that they are \emph{definitionally} or \emph{intensionally} equal. This is denoted
by $d \equiv e$.
\end{mydef}

From the expression definition rules, we can see a few equalities arising:
\begin{eqnarray*}
  e & \equiv & ((x)e)\;x \\
  e_i & \equiv & (e_1, \cdots, e_n).i
\end{eqnarray*}

As probably noticed, we mentioned \emph{expressions of suitable arity}, but did not explain
what arity means. The notion of arity is somewhat different from what one would expect, it can
be seen as a \emph{meta-type} of the expression, and indicates which expressions can be combined
together.

\newcommand{\arzero}{\textbf{0}}
\newcommand{\ararr}{\twoheadrightarrow}
\newcommand{\armul}{\otimes}

Expressions are divided in a couple classes. It is either \emph{combined}, from which we
might select components from it, or it is \emph{single}. In addition, an expression
can be either \emph{saturated} or \emph{unsaturated}. A single saturated expression
have arity $\arzero$, therefore, neither selection nor application can be performed.
Unsaturated expressions on the other hand, have arities of the form $\alpha \ararr \beta$,
where $\alpha$ and $\beta$ are themselves arities. The informal meaning of such arity is \emph{"give me an expression
of arity $\alpha$ and I give an expression of arity $\beta$"}, just like normal Haskell types. 

For example, the built in $succ$ has arity $\arzero \ararr \arzero$; the list constructor
$cons$ has arity $(\arzero \armul \arzero) \ararr \arzero$, since it takes two expressions
of arity $\arzero$ and returns an expression. We will not go into much more detail 
on arities. It is easy to see how they are inductively defined. We refer the reader
to chapter 3 of \cite{nords90}.

Evaluation of expressions in the theory of types is performed in a lazy fashion, the semantics
being based in the notion of \emph{canonical expression}. These canonical expressions are the values
of programs and, for each set, they have different formation conditions. The common property is
that they must be closed and saturated. It is closely related to the weak-head normal form concept in
lambda calculus, as illustrated in table \ref{table:expr_evaluation_state}.

\begin{center}
\begin{table}[h]
\begin{tabular}{p{3cm} p{3cm} p{3cm} p{3cm}}
  Canonical & Noncanonical & Evaluated & Fully Evaluated \\  
  $12$ & $fst\;\langle a , b \rangle$ & $succ\;zero$ & $true$ \\
  $false$ & $3 \times 3$ & $succ\;(3 + 3)$ & $succ\;zero$ \\
  $(\lambda x . x)$ & $(\lambda x . snd\;x) p$ & $cons(4, app(nil, nil))$ & $4$
\end{tabular}
\caption{Expression Evaluation State}
\label{table:expr_evaluation_state}
\end{table}
\end{center}

\subsection{Judgement Forms}
Standing on top of the basic constructors of the theory of types, we can start to discuss
what kind of judgement forms we can express and derive. In fact, Agda boils down to a tool
that does not allow us to make incorrect derivations. Type theory provides us with derivational 
rules to discuss the validity of judgements of the form given in table \ref{table:judgement_forms}

\begin{center}
\begin{table}[h]
\begin{tabular}{c l l}
       & Sets & Propositions \\ \hline
  \textit{(i)}   & $A$ is a set. & $A$ is a proposition. \\
  \textit{(ii)}  & $A$ and $B$ are equal sets. & $A$ and $B$ are equivalent propositions. \\
  \textit{(iii)} & $a$ is an element of a set $A$. & $a$ is a proof of $A$. \\
  \textit{(iv)}  & $a$ and $b$ are equal elements of a set $A$. & $a$ and $b$ are the same proof. \\  
\end{tabular}
\caption{judgement Forms}
\label{table:judgement_forms}
\end{table}
\end{center}

When reading a set as a proposition, we might simplify $(iii)$ to \emph{$A$ is true}, disregarding
the proof. What matters is the existence of the proof.

But then, let's take $(i)$ as an example. What does it means to be a set? To know that $A$ is a set is
to know how to form the canonical elements of $A$ and under what conditions two canonical
elements are equal. Therefore, to construct a set, we need to give a syntactical description
of its canonical elements and provide means to decide whether or not two canonical elements
are equal. Let us take another look at the Peano Naturals:

\Agda{Basic}{NAT}

In fact defines a \D{Set}, whose canonical elements are either \IC{zero} or
\IC{succ} applied to something. Equality in \D{Nat} is indeed decidable, therefore we have a set
in the theory of types sense.

The third form of judgement might also be slightly tricky. Given that $A$ is a set, 
to know that $a$ is an element of $A$ amounts to knowing that, when evaluated, $a$ yields a canonical element in $A$
as value. Making the parallel to Agda again, this is what allows one to pattern match on terms.

To know that two sets are equal is to know that they have the same canonical elements, and
equal canonical elements in $A$ are also equal canonical elements in $B$. Two elements are
equal in a set when their evaluation yields equal canonical elements.

For a proper presentation of the theory of types, we should generalize these judgement forms
to cover hypotheses. As this is not required for 
a stable understanding of Agda. we refer the reader to Martin-L\"{o}f's thesis \cite{lof84,lof85}.

\subsection{General Rules}

A thorough explanation of the rules for the theory of types is outside the scope of this section. 
This is just a simple illustration that might help unfamiliar readers to grasp Agda with
some comfort, so that they get a taste of what is happening behind the curtains. For this, it is enough 
to present the general notion of rule, their syntax and give a simple example.

In the theory of types there are some general rules regarding equality and substitution, 
which can be justified by the defined semantics. Then, for each set forming operation, there
are rules for reasoning about the set and its elements. These foremost classes of rules
are divided into four kinds:

\begin{enumerate} %[i)]
  \item \emph{Formation} rules for $A$ describe the conditions under which $A$ is a set and
        when another set $B$ is equal to $A$.
        
  \item \emph{Introduction} rules for $A$ are used to construct the canonical elements. They
        describe how they are formed and when two canonical elements of $A$ are equal.
        
  \item \emph{Elimination} rules act as a \emph{structural induction principle} for elements of $A$.
        They allow us to prove propositions about arbitrary elements.
        
  \item \emph{Equality} rules gives us the equalities which are associated with $A$.
\end{enumerate}

\newcommand{\isset}[1]{#1 \; \text{set}}
\newcommand{\withhip}[2]{#2 \; [ #1 ] }
\newcommand{\BB}{\mathbb{B}}
\newcommand{\ite}[3]{(#1 \; ? \; #2 \; : \; #3)}
The syntax will follow a natural deduction style. For example: 
\[
  \infer{\isset{\Pi(A , B)}}{\isset{A} & \withhip{x \in A}{\isset{B(x)}}}
\]
The rule represents the formation rule for $\Pi$, and can be applied whenever $A$ and $B(x)$ are sets,
assuming that $x \in A$. judgements may take the form $p = a \in A$, meaning 
that if we compute the program $p$, we get the canonical element $a \in A$ as a result.

So far so good. But how do we use all these rules and interpretations? Let us to take the
remaining fog out of how Martin-L\"{o}f's theory \emph{is} Agda with the next example. For this,
we need the set of booleans and the set of intentional equality.

\subsubsection{Boolean Values}

The set $\{ true, false \}$ of boolean values is nothing more than an enumeration set.
There are generic rules to handle enumeration sets. For simplicity's sake we are going to use an instantiated
version of such rules.

\[ 
\begin{array}{c c c}
    \infer[\BB_{form}]{ \isset{\BB} }{} 
  & \infer[\BB_{intro_1}]{true \in \BB}{}
  & \infer[\BB_{intro_2}]{false \in \BB}{}
\end{array}
\]
\[
    \infer[\BB_{eq_1}]
          {\ite{true}{c}{d} = c \in C(true)}
          {\withhip{b \in \BB}{\isset{C(b)}}
          & c \in C(true)
          & d \in C(false)
          }
\] 
\vspace{2mm}
\[
    \infer[\BB_{eq_2}]
          {\ite{false}{c}{d} = d \in C(false)}
          {\withhip{b \in \BB}{\isset{C(b)}}
          & c \in C(true)
          & d \in C(false)
          }
\] 
\vspace{2mm}
\[
    \infer[\BB_{elim}]
          {\ite{b}{c}{d} \in C(b)}
          { b \in \BB
          & \withhip{v \in \BB}{\isset{C(v)}}
          & c \in C(true)
          & d \in C(false)
          }
\]
\vspace{2mm}

\subsubsection{Intensional Equality}

The set $Id(A, p, a)$, for $Id$ a primitive constant (with arity $\arzero \armul \arzero \armul \arzero \ararr \arzero$),
represents the judgement $p = a \in A$. Put into words, it means that the program $p$ evaluates to $a$,
which is a canonical element of $A$. The elimination and equality rules for $Id$ will not be presented here.

\[
  \infer[Id_{form}]
        {\isset{Id(A, a, b)}}
        { \isset{A}
        & a \in A
        & b \in A
        }
\]
\[
\begin{array}{c c}
  \infer[Id_{intro_1}]
        {id(a) \in Id(A, a, a)}
        {a \in A}
&
  \infer[Id_{intro_2}]
        {id(a) \in Id(A, a, b)}
        {a = b \in A}
\end{array}
\]

From the introduction rules, we should be able to tell which are the canonical elements of $Id$.
As an exercise, the reader should compare the set $Id$ with Agda's \emph{Relation.Binary.PropositionalEquality}\footnote{%
%%%% BEGINFOOTNOTE
Available from the Agda Standard Library: https://github.com/agda/agda-stdlib
%%%% ENDFOOTNOTE
}.

\subsection{Epilogue}

At this point we have briefly discussed the Curry-Howard isomorphism in both the simply-typed
version and the dependently typed flavor. We have presented (the very surface of) the theory
of types and which kind of logical judgements we can handle with it. But how does all this connect
to Agda and verification in general?

Software Verification with a functional language is somewhat different from doing the same with a imperative language. 
If one is using C, for example, one would write the program and annotate it with logical expressions.
They are twofold. We can use pre-conditions, post-conditions and invariants to deductively verify a program
(these are in fact a variation of Hoare's logic). Or, we can use Software Model Checking, proving
that for some bound of traces $k$, our formulas are satisfied.

When we arrive at the functional realm, our code is correct \emph{by construction}. The type-system
provides both the \emph{annotation language} and you can construct a program that either type-checks,
therefore respects its specification, or does not pass through the compiler. Remember that Agda's
type system is the intensional variant of Martin-L\"{o}f's theory of types, which was presented above
in this chapter.

Let us now prove that, for any set $A$ and $a \in A$, given a $b \in \BB$ we have
that $\ite{b}{a}{a} = a$. Translating it to a more formal version, we want to prove that:
\[
  \withhip{b \in \BB, a \in A}{Id(A, \ite{b}{a}{a}, a)} \text{ is inhabited.}
\]

Well, it suffices to show the existence of some element in the aforementioned set.

Let us denote the set $Id(A, \ite{x}{a}{a}, a)$ by $\phi(x)$, for any given set $A$ and
element $a \in A$. The (partially ommited) derivation follows.

\[
\infer[\BB_{elim}]
      {\ite{b}{id(a)}{id(a)} \in \phi(b) }
      { b \in \BB 
      & \infer*{\withhip{v \in \BB}{\isset{\phi(v)}}}{}
%              { \infer[Id_{form}]
%                      {\isset{Id(A, \ite{b}{a}{a}, a)}}
%                      { \isset{A}
%                      & \ite{b}{a}{a} \in A
%                      & a \in A
%                      }
%              }
      & \infer[Id_{intro}]
              {id(a) \in \phi(true)}
              { \infer[\BB_{eq_1}]
                      {\ite{true}{a}{a} = a \in A}
                      {\isset{A}
                      & a \in A
                      } 
              }
      & \infer[Id_{intro}]
              {id(a) \in \phi(false)}
              { \infer[\BB_{eq_2}]
                      {\ite{false}{a}{a} = a \in A}
                      {\isset{A}
                      & a \in A
                      } 
              }
      }
\] 


Now, take a look at the same proof, encoded in Agda without anything from the standard library.
Note how some rules (like $\BB_{elim}$) are built into the language, as pattern-matching, for instance.

\Agda{MartinLof}{proof}

The Agda snippet above is fairly more readable than the actual derivation of our example lemma.
Understanding how one writes programs and very general proofs in the same language can be tough.
We hope to have exposed how Agda uses Martin-L\"{o}f's theory of types in a very clever way, 
providing a very expressive logic for theorem proving. The programming
part of Agda is directly connected to last chapter's Curry-Howard Isomorphism.
This concludes a big portion of the background needed for understanding the rest of this dissertation.


  
\chapter{Relational Algebra in Agda}
\label{chap:relationalalgebrainagda}
\begin{TODO}
  \item frame this thing
\end{TODO}

We're taking Relational Algebra\cite{Bird97} as our main case study for a couple of reasons. One of the most
important beeing it's expressive power and it's unquestionable advantage as a framework for
reasoning about software, in an equational fashion. Relational Algebra is in fact a discipline that
allows us to speak of software engineering through unambiguous equations, giving a definite meaning
to the \emph{engineering} part of software engineering.

In this chapter we will introduce a basic encoding for relations, then we'll discuss several 
further options that would change the handling of equality, which turned out to be a very
delicate matter. The problem imposed by Agda's (intensional) equality is that it implies
convertibility. As we shall see later, we don't really care about convertibility of relations,
but that they're inhabited at the same time. We can borrow some notions from Homotopy Type Theory\cite{hottbook}
in order to hardcode a proof irrelevance notion, but this adds a very complex layer of boilerplate
code for the end-user. Another option is to postulate some axioms that will allow us to \emph{trick}
Agda's equality, beeing always carefull not to introduce a contradiction. 
There are a lot of options for this later case too, we shall discuss a few of them.

\section{Encoding Relations}

Unlike most programming languages, an encoding of Relational Algebra in a dependently typed 
language allows one to truly see the advantage of dependent types in action. Most of the definitions
happens at the \emph{type level} (remember that there is no difference between types and values in Agda,
this is just a mnemonic to help understanting the \emph{functions}). The encoding presented
here is based on \cite{Jansson09}.

Long story short, a binary relation $R$ of type $A \rightarrow B$ can be thought of in terms of several mathematical objects.
The usual definition is to say that $R \subseteq A \times B$, where $\cdot\times\cdot$ is the cartesian product of sets.
In fact, $R$ contains pairs or related elements whose first component is of type $A$ and second component is of type $B$,
note that it is slightly more general than the concept of a function, where we can have $b_1\;R\;a$ and $b_2\;R\;a$, which means
that $(a, b_1) \in R$ and $(a, b_2) \in R$, as a perfectly valid relation, but not a function, since $a$ would be mapped to two different values, $b_1$ and $b_2$.

\newcommand{\powerset}{\mathcal{P}}
Another way of speaking about relations, though, is to consider functions of type $A \rightarrow \powerset B$.
If our previous $R$ was in fact a function $f$, we would then have $f\; a = \{b_1, b_2\}$. For the more
mathematically inclined reader, the arrows $A \rightarrow \powerset B$ in the category \catname{Sets}, of sets and
functions, correspond to arrows $A \rightarrow B$ in the category \catname{Rel}, of sets and relations. For this matter,
we actually call \catname{Rel} the Kleisli Category for the monad $\powerset$. 
In fact, we can even define the powerset transpose of a relation $R$, which is a function that for each $a$
returns the subset of $B$ that is related to $a$ trough $R$.
\[ (\Lambda\; R)\; a = \{ b \in B \;\mid\; b\;R\;a \} \]

For our Agda encoding of Relations, we shall use a slight modification of the $\powerset$ approach.
Let's begin, in fact, by encoding set theoretic notions. The most important of all beeing undoubtly
the membership notion $\cdot \in \cdot$. One way of encoding a subset of a set $A$ is using a function $f$
of type $A \rightarrow Bool$, the subset is obtained by $\{\; a \in A\; |\; f\;a\; \}$. Yet, in Agda, this would
force that we deal only with decidable domains, which is not a problem if everything is finite, but
for infinite domains this would not work.

Another option, which is the one used in \cite{Jansson09}, is to use a function $f$
of type $A \rightarrow \D{Set}$ to encode a subset of $A$. Remember that \D{Set} is
the type of types in Agda. Although not very intuitive, this is much more expressive than the 
last option and the induced subset would be defined by $\{\; a \in A\; | f\;a\;\text{is inhabited }\}$,
which would turn out to be:

\Agda{RelationsCore}{subset-def}

Extending from sets to binary relations is a very simple task. Besides the canonical steps,
we'll also swap the arguments for a relation, following what was done in \cite{Jansson09} and
keeping the syntax closer to what one would write on paper, since we usually writes a relational
statement from left to right, that is, $y\;R\;x$ means that $(x, y) \in R$.
\begin{eqnarray*}
  \powerset (A \times B) & = & \powerset (B \times A)\\
                         & = & B \times A \rightarrow \D{Set}\\
                         & = & B \rightarrow A \rightarrow \D{Set}
\end{eqnarray*}

Here we present the the base encoding for relations in Agda, together with a few constructs
to help in their definition and to illustrate usage.

\Agda{RelationsCore}{relation-def}

And the operations are defined just as we would expect them. Note that \D{\_⊎\_} represents a disjoin union of sets
(equivalent to Either in Haskell) and \D{\_×\_} represents the usual cartesian product. The function
lifting might look like the less intuitive construction, but it's a very simple one. $fun\;f$ is a relation,
therefore it takes a $b$ and a $a$ and should return a type that is inhabited if and only if $(a, b) \in fun\;f$,
or, to put it another way, if $b = f\;a$, almost like it's textbook definition.

\begin{TODO}
  \item Fix the definition of composition in the library. Explain it here.
  \item Add a few universals. Explain how we're taking the road in the "wrongway".
        We first give a definition then we prove it satisfies the universals, instead
        of using universals as definitions.
  \item Explain Agda's propositional equality?
\end{TODO}

\section{Relational Equality}

A notion of convertibility is paramount to any form of rewriting. In out scenario, whenever
two relations are equal, we can substitute them back and forth in a given equation. There are
a couple different, but equivalent notions of equality. The simplest one is borrowed from set
theory.\\

\begin{mydef}[Relation Inclusion]
Let $R$ and $S$ be suitably typed binary relations, we say that $R \subseteq S$ 
if and only if for all $a , b$, if $b\;R\;a$ then $b\;S\;a$.\\
\end{mydef}

\begin{mydef}[Relational Equality]
Let $R$ and $S$ be as before, we say that $R \releq S$ if and only if
$R \subseteq S$ and $S \subseteq R$.\\
\end{mydef}

\begin{lemma}
Relation Inclusion is a partial order, that is, $\subseteq$ is reflexive, transitive and anti-symmetric.
And $\releq$ is an equivalence relation (reflexive, transitive and symmetric).\\
\end{lemma}

So far so good, at first sight one would believe that there are no problems with such definitions
(and in fact there won't be any problem as long as we keep away from Agda, but then it wouldn't be fun, right?). 
Ignore the boilerplate code, we wraped both definitions
as datatypes in Agda to prevent expansion when using reflection.

\Agda{RelationsCore}{subrelation}

Allow us to take a deeper look into what this definition say, though. 
Let $R : \D{Rel}\;A\;B$, for $A$ and $B$ \D{Set}s, $a : A$ and $b : B$. When we
state $R\;b\;a$ we're obtaining a \D{Set}, and an element $r : R\;b\;a$ represents
a proof that $b$ is related to $a$ trough $R$. We have no interest in the contents of
such proof though, it's existence is what matters. In more formal terms, we would like
that $\D{Rel}$ returned a proof irrelevant set. In Coq this would be \texttt{\small Prop}, but
Agda has no set of propositions.

We can see such problem comming to the surface if we try to use Agda's propositional equality 
instead of $\releq$ (and hence the name distinction):

\Agda{RelationsCore}{releq-is-not-propeq}

We have to use functional extensionality anyway (here, disguised as \F{rel-ext}). But the 
type of our goal is $R\;b\;a \equiv S\;b\;a$ and the proof get stuck since set equality is
undecidable \warnme{is it? Be sure!}. We infact just want $R$ and $S$ to be inhabited at
the same time.

So, it's clear that we also need some encoding of $\releq$:

\Agda{RelationsCore}{releq}

But this definition is not substitutive in Agda. Therefore we need to lift it to
a substitutive definition. In fact, it is safe to lift it to the standard propositional equality.
The advantages of doing so are that we win all of the functionality to work with $\equiv$ for free.

The solution we're currently adopting relies on using yet another disguised version of
function extensionality (here as powerset transpose extensionality) 
and postulating the proof-irrelevant notion of set equality (as in Set Theory):

\Agda{RelationsCore}{releq-lifting}

The powerset transpose postulate is pretty much function-extensionality with some syntatic sugar, and it is
known to not introduce any contractiction. But $ℙ-ext$ on the other hand, have to be justified.
The reason for such postulate is just because somewhere in our library, we gotta have a notion
of proof-irrelevance to be able to state relational equality. As we'll discuss in the next section,
there are a couple places were we can insert this notion, but postulating it at the subsets ($ℙ$) level is
by far the easier to handle and introduces the least amount of boilerplate code. (Remember that
if $a$ is a subset of $A$, that is $a : ℙ A$, stating $a\;x$ is stating $x \in a$, for any $x : A$.).

There are a few other options of achieving a substitutive behavior for $\releq$, they all
rely on it's lifting to propositional equality, though. As we shall present in the next 
sections.

\begin{TODO}
  \item Indirect equality somewhere?
\end{TODO}

\subsection{Hardcoded Proof Irrelevance}

  We mentioned before this idea of a proof irrelevant set, we also mentioned that we postulated
  such notion, which is not the ideal thing to do in Agda. We want to keep our postulates to an
  absolute minimum.
  
  In order to formally talk about proof irrelevance we need some concepts from Homotopy Type Theory\cite{hottbook}.
  HoTT is an immense and complex newly founded field of Mathematics, and we are not going to explain
  it in detail. The general idea is to use abstract Homotopy Theory to interpret types. In normal
  type theory, a statement $a : A$ is interpreted as $a$ is an element of the set $A$. In HoTT, however,
  we say that $a$ is a point in space $A$. Objects are seen as points in a space and the type $a = b$ is seen as a path from $a$ to $b$.
  It's obvious that for all $a$ there is a path from $a$ to $a$, therefore $a = a$, giving us reflexivity.
  If we have a path from $a$ to $b$, we also have one from $b$ to $a$, resulting in symmetry.
  \emph{Glueing} of paths together corresponds to transitivity. So we have an equivalence relation $=$.
  
  Even more important, if we have a proof of $P a$ and $a = b$, for a predicate $P$, we can
  transport this proof along the path $a = b$ and arrive at a proof of $P b$. This corresponds to
  the substitutive behavior of standard equality. Note though, that in classical mathematics,
  once an equality $x = y$ has been proved, we can just switch $x$ and $y$ whenever we want. 
  In HoTT, however, there might be more than one path from $x$ to $y$, and they might yield different
  results, it is important to state which path is beeing \emph{walked} when we use substitutivity.
  
  The first important notion is that of homotopy. Traditionally, we take that two functions are
  the same if they agree onall inputs. A homotopy between two functions is very close to that
  classical notion:
  
  \Agda{ProofIrrel}{homotopy-def}
  
  It's worth mentioning that \F{\_\~\_} is an equivalence relation, although we ommit the proof.
  We follow by defining an equivalence, which can be seen an isomorphism notion:
  
  \Agda{ProofIrrel}{isequiv-def}
  
  That is, $f$ is an equivalence if there exists a $g$ such that $g$ is a left and right-inverse
  of $f$.
  
  Now we can state when two types are equivalent. And for the reader familiar with algebra, it is
  not very far from the usual isomorphism-based equivalence notion. As expected, univalence is
  also an equivalence relation.
  
  \Agda{ProofIrrel}{univalence-def}
  
  Following the common pracice when encoding HoTT in Agda, we have to postulate the Univalence
  axiom, which in short says that univalence and equivalence coincide:
  
  \Agda{ProofIrrel}{univalence-axiom}
  
  Now, our job becomes much easier, and it suffices to show that if two relations are
  mutually included, then they are univalent.
  
  \subsubsection{Mere Propositions}
  
  As we mentioned earlier, proof irrelevance is a desired property in most systems. In HoTT,
  one distinguish between mere propositions and other types, where mere propositions
  are defined by:
  
  \Agda{ProofIrrel}{isprop-def}
  
  This allows us to state some very usefull properties, which allows us to handle propositions
  as both true or false, depending on wether or not they're inhabited. These corresponds to
  lemma 3.3.2 in \cite{hottbook}.
  
  \Agda{ProofIrrel}{lemma-332}\\
  \Agda{ProofIrrel}{not-lemma-332}
  
  Which are both provable in Agda, but the proofs are ommited here.
  
  \subsubsection{Adding Relations to the mix}
  
  Now, to exploit such finer treatment of equality in our favor, we need to add
  relation and a few other details to the mix. We'll keep the relation definition
  as before, and push to the user the responsability of proving that his relations
  are both mere propositions and decidable.
  
  This can be easily done with Agda's instance mechanism:
  
  \Agda{ProofIrrel}{instances}
  
  This will treat both records as typeclasses in the Haskell sense. Now, for talking about
  subrelations they must be an instance of \D{IsProp}, and whenever we wish to use anti-symmetry
  they must also be an instance of \D{IsDec}, and it turns out that anti-symmetry is now
  provable as long as we assume relational extensionality.
  
  \Agda{ProofIrrel}{subrel-def}\\
  \Agda{ProofIrrel}{subrel-antisym}
  
  Although we could arrive at the result we desired with a minimal number of postulates (relational extensionality
  and univalence axiom), the user was heavily punished for when he wants to define a relation, not to
  say that decidability will give problems once relational composition enters the stage. For this reasons
  we chose not to adopt this solution, even though it's more formal than what we have.

\subsubsection{Custom Universes}


  We could remove the \D{IsProp} record from our code, if we gave relations a bit more structure,
  and, prove that every object in this new (more structured) world is a mere proposition.
  One good option would be to encode a universe $U$ of mere propositions and have relations
  defined as $B \rightarrow A \rightarrow U$. This extra structure allows us to prove proof irrelevance
  for all $u : U$ (which holds by construction, once $u$ is a mere proposition), but only let's us define
  relations over $U$, which is less expressive than $Set$. The additional code boilerplate is also big,
  once we have to define a language and all operations that we'll need to perform with it.
  
  A universe and it's interpretation are defined as:
  
  \Agda{Universes}{data-U}
  
  Where $\mid\mid\_\mid\mid$ is a propositional truncation. That is, for every type $A$,
  there is a type $\mid\mid A \mid\mid$ with two constructors: (i) for any $x : A$ we have
  $\mid x \mid : \mid\mid A \mid\mid$; (ii) for any $x , y : \mid\mid A \mid\mid$, we have
  $x \equiv y$. This is also called smashing sometimes. Not every type constructor 
  preserves mere propositions. A simple example is with the coproduct
  $1 + 1$. Even though $1$ is a mere proposition, $1 + 1$ is not, since the elements of such type
  contain also information about which injection was used; We need to smash this information out
  if we want to keep with mere propositions.
  
  It turns out that we're just defining the logic we'll need to define relations, but this
  structure is very helpful! Now we can prove that given $u : U$, $\sharp u$ is a mere proposition.
  
  \Agda{Universes}{uprop-decl}
  
  So far so good! We just removed one instance from the user and proving decidability can be very
  straight forward (but in a few, rare, complicated cases)! Well, the changes were not only for the best.
  A new, minor, problem is the verbosity introduced by $U$, everything is harder to write and read.
  But there's a bigger situation happening here. If we look at the existential quantifier defined in $U$,
  it's witnesses must also come from $U$. This can be very restrictive once we start using relational
  composition (which is defined in terms of an existential).




  \section{State of the Art}
  \label{sec:relalg_sota}
  The current developments within Relational Algebra in Agda vary a lot, mainly due to
different goals. One has a plethora of options when developing a library. It is therefore
important to weight the options with the goals. As far as we know, at the time
of writing this thesis, there was no library that fitted our specific goals, which
are: (A) An expressive encoding of relational algebra; and (B) easy to handle in the meta-level,
for automatic rewritting.

The library constructed by Kahl, \cite{RATHAgda}, is not specific to Relational Algebra, but
instead he chose to construct a library for Category Theory, from which one can use
the specific category of relations, \catname{Rel}. The theories in RATH-Agda are intended
for a high-level programming, not so much focused on theorem proving. We can see that
the equality problem we ran into is not handled in Kahl's library. RATH-Agda has both a equality up to
isomorphism and intensional equality, not interchangeable. Later on, the library
defines the relational operations on top of the categorical basis, also provided by the library.
Relational Equality is defined by mutual inclusion, and, since no rewriting was intended,
substitutivity of relational equality is not provided. Another problem we would run into
if we had used RATH-Agda is the quoted representation of terms. In Kahl's library every
construction is defined as a top-level function, therefore it is expanded upon quoting. 

Another interesting library is the one developed by Jansson, Mu and Ko at \cite{Jansson09}. 
Their goal, however, still is very different from both RATH-Agda and this project.
Mu, et al., are focussing on deriving programs given specifications, which is a big merit of
functional programming. Indeed, they explore the equational reasoning of programs in order to keep
\emph{refining} programs. They also avoid the non-extensional equality problem by defining
another equality type and proving its substitutiveness whenever necessary. Their main focus
is on subrelation-reasoning, though.


  \section{Encoding Relations}
  \label{chap:relationalalgebra}
  Unlike most programming languages, an encoding of Relational Algebra in a dependently typed 
language allows one to truly see the advantage of dependent types in action. Most of the definitions
happens at the \emph{type level} (remember that there is no difference between types and values in Agda,
this is just a mnemonic to help understanding the \emph{functions}). The encoding presented
here is based on \cite{Jansson09}, with most of it's significant differences being due to
extensional equality.

Long story short, a binary relation $R$ of type $A \rightarrow B$ can be thought of in terms of several mathematical objects.
The usual definition is to say that $R \subseteq A \times B$, where $\cdot\times\cdot$ is the cartesian product of sets.
In fact, $R$ contains pairs or related elements whose first component is of type $A$ and second component is of type $B$,
note that it is slightly more general than the concept of a function, where we can have $b_1\;R\;a$ and $b_2\;R\;a$, which means
that $(a, b_1) \in R$ and $(a, b_2) \in R$, as a perfectly valid relation, but not a function, since $a$ would be mapped to two different values, $b_1$ and $b_2$.

\newcommand{\powerset}{\mathcal{P}}
Another way of speaking about relations, though, is to consider functions of type $A \rightarrow \powerset B$.
If our previous $R$ was in fact a function $f$, we would then have $f\; a = \{b_1, b_2\}$. For the more
mathematically inclined reader, the arrows $A \rightarrow \powerset B$ in the category \catname{Sets}, of sets and
functions, correspond to arrows $A \rightarrow B$ in the category \catname{Rel}, of sets and relations. For this matter,
we actually call \catname{Rel} the Kleisli Category for the monad $\powerset$. 
In fact, we can even define the powerset transpose of a relation $R$, which is a function that for each $a$
returns the subset of $B$ that is related to $a$ trough $R$.
\[ (\Lambda\; R)\; a = \{ b \in B \;\mid\; b\;R\;a \} \]

For our Agda encoding of Relations, we shall use a slight modification of the $\powerset$ approach.
Let's begin, in fact, by encoding set theoretic notions. The most important of all being undoubtedly
the membership notion $\cdot \in \cdot$. One way of encoding a subset of a set $A$ is using a function $f$
of type $A \rightarrow Bool$, the subset is obtained by $\{\; a \in A\; |\; f\;a\; \}$. Yet, in Agda, this would
force that we deal only with decidable domains, which is not a problem if everything is finite, but
for infinite domains this would not work.

Another option, which is the one used in \cite{Jansson09}, is to use a function $f$
of type $A \rightarrow \D{Set}$ to encode a subset of $A$. Remember that \D{Set} is
the type of types in Agda. Although not very intuitive, this is much more expressive than the 
last option and the induced subset would be defined by $\{\; a \in A\; | f\;a\;\text{is inhabited }\}$,
which would turn out to be:

\Agda{RelationsCore}{subset-def}

Extending from sets to binary relations is a very simple task. Besides the canonical steps,
we'll also swap the arguments for a relation, following what was done in \cite{Jansson09} and
keeping the syntax closer to what one would write on paper, since we usually writes a relational
statement from left to right, that is, $y\;R\;x$ means that $(x, y) \in R$.
\begin{eqnarray*}
  \powerset (A \times B) & = & \powerset (B \times A)\\
                         & = & B \times A \rightarrow \D{Set}\\
                         & = & B \rightarrow A \rightarrow \D{Set}
\end{eqnarray*}

Here we present the the base encoding for relations in Agda, together with a few constructs
to help in their definition and to illustrate usage.

\Agda{RelationsCore}{relation-def}

And the operations are defined just as we would expect them. Note that \D{\_⊎\_} represents a disjoint union of sets
(equivalent to Either in Haskell) and \D{\_×\_} represents the usual cartesian product. The function
lifting might look like the less intuitive construction, but it's a very simple one. $fun\;f$ is a relation,
therefore it takes a $b$ and a $a$ and should return a type that is inhabited if and only if $(a, b) \in fun\;f$,
or, to put it another way, if $b = f\;a$, almost like it's textbook definition.

\begin{TODO}
  \item Fix the definition of composition in the library. Explain it here.
  \item Add a few universals. Explain how we're taking the road in the "wrongway".
        We first give a definition then we prove it satisfies the universals, instead
        of using universals as definitions.
  \item Explain Agda's propositional equality?
\end{TODO}

\section{Relational Equality}

A notion of convertibility is paramount to any form of rewriting. In out scenario, whenever
two relations are equal, we can substitute them back and forth in a given equation. There are
a couple different, but equivalent notions of equality. The simplest one is borrowed from set
theory.\\

\begin{mydef}[Relation Inclusion]
Let $R$ and $S$ be suitably typed binary relations, we say that $R \subseteq S$ 
if and only if for all $a , b$, if $b\;R\;a$ then $b\;S\;a$.\\
\end{mydef}

\begin{mydef}[Relational Equality]
Let $R$ and $S$ be as before, we say that $R \releq S$ if and only if
$R \subseteq S$ and $S \subseteq R$.\\
\end{mydef}

\begin{lemma}
Relation Inclusion is a partial order, that is, $\subseteq$ is reflexive, transitive and anti-symmetric.
And $\releq$ is an equivalence relation (reflexive, transitive and symmetric).\\
\end{lemma}

So far so good, at first sight one would believe that there are no problems with such definitions
(and in fact there won't be any problem as long as we keep away from Agda, but then it wouldn't be fun, right?). 
Ignore the boilerplate code, we wrapped both definitions
as datatypes in Agda to prevent expansion when using reflection.

\Agda{RelationsCore}{subrelation}

Allow us to take a deeper look into what this definition say, though. 
Let $R : \D{Rel}\;A\;B$, for $A$ and $B$ \D{Set}s, $a : A$ and $b : B$. When we
state $R\;b\;a$ we're obtaining a \D{Set}, and an element $r : R\;b\;a$ represents
a proof that $b$ is related to $a$ trough $R$. We have no interest in the contents of
such proof though, it's existence is what matters. In more formal terms, we would like
that $\D{Rel}$ returned a proof irrelevant set. In Coq this would be \texttt{\small Prop}, but
Agda has no set of propositions.

We can see such problem coming to the surface if we try to use Agda's propositional equality 
instead of $\releq$ (and hence the name distinction):

\Agda{RelationsCore}{releq-is-not-propeq}

We have to use functional extensionality anyway (here, disguised as \F{rel-ext}). But the 
type of our goal is $R\;b\;a \equiv S\;b\;a$ and the proof get stuck since set equality is
undecidable. We in fact just want $R$ and $S$ to be inhabited at
the same time.

So, it's clear that we also need some encoding of $\releq$:

\Agda{RelationsCore}{releq}

But this definition is not substitutive in Agda. Therefore we need to lift it to
a substitutive definition. In fact, it is safe to lift it to the standard propositional equality.
The advantages of doing so are that we win all of the functionality to work with $\equiv$ for free.

The solution we're currently adopting relies on using yet another disguised version of
function extensionality (here as powerset transpose extensionality) 
and postulating the proof-irrelevant notion of set equality (as in Set Theory):

\Agda{RelationsCore}{releq-lifting}

The powerset transpose postulate is pretty much function-extensionality with some syntactic sugar, and it is
known to not introduce any contradiction. But $ℙ-ext$ on the other hand, have to be justified.
The reason for such postulate is just because somewhere in our library, we gotta have a notion
of proof-irrelevance to be able to state relational equality. As we'll discuss in the next section,
there are a couple places were we can insert this notion, but postulating it at the subsets ($ℙ$) level is
by far the easier to handle and introduces the least amount of boilerplate code. (Remember that
if $a$ is a subset of $A$, that is $a : ℙ A$, stating $a\;x$ is stating $x \in a$, for any $x : A$.).

There are a few other options of achieving a substitutive behavior for $\releq$,  
as we shall present in the next sections.

\subsection{Hardcoded Proof Irrelevance}

  We mentioned before this idea of a proof irrelevant set, we also mentioned that we postulated
  such notion, which is not the ideal thing to do in Agda. We want to keep our postulates to an
  absolute minimum.
  
  In order to formally talk about proof irrelevance we need some concepts from Homotopy Type Theory\cite{hottbook}.
  HoTT is an immense and complex newly founded field of Mathematics, and we are not going to explain
  it in detail. The general idea is to use abstract Homotopy Theory to interpret types. In normal
  type theory, a statement $a : A$ is interpreted as $a$ is an element of the set $A$. In HoTT, however,
  we say that $a$ is a point in space $A$. Objects are seen as points in a space and the type $a = b$ is seen as a path from $a$ to $b$.
  It's obvious that for all $a$ there is a path from $a$ to $a$, therefore $a = a$, giving us reflexivity.
  If we have a path from $a$ to $b$, we also have one from $b$ to $a$, resulting in symmetry.
  \emph{Gluing} of paths together corresponds to transitivity. So we have an equivalence relation $=$.
  
  Even more important, if we have a proof of $P a$ and $a = b$, for a predicate $P$, we can
  transport this proof along the path $a = b$ and arrive at a proof of $P b$. This corresponds to
  the substitutive behavior of standard equality. Note though, that in classical mathematics,
  once an equality $x = y$ has been proved, we can just switch $x$ and $y$ whenever we want. 
  In HoTT, however, there might be more than one path from $x$ to $y$, and they might yield different
  results, it is important to state which path is being \emph{walked} when we use substitutivity.
  
  The first important notion is that of homotopy. Traditionally, we take that two functions are
  the same if they agree on all inputs. A homotopy between two functions is very close to that
  classical notion:
  
  \Agda{ProofIrrel}{homotopy-def}
  
  It's worth mentioning that \F{\_\~{}\_} is an equivalence relation, although we omit the proof.
  We follow by defining an equivalence, which can be seen an isomorphism notion:
  
  \Agda{ProofIrrel}{isequiv-def}
  
  That is, $f$ is an equivalence if there exists a $g$ such that $g$ is a left and right-inverse
  of $f$.
  
  Now we can state when two types are equivalent. And for the reader familiar with algebra, it is
  not very far from the usual isomorphism-based equivalence notion. As expected, univalence is
  also an equivalence relation.
  
  \Agda{ProofIrrel}{univalence-def}
  
  Following the common practice when encoding HoTT in Agda, we have to postulate the Univalence
  axiom, which in short says that univalence and equivalence coincide:
  
  \Agda{ProofIrrel}{univalence-axiom}
  
  Now, our job becomes much easier, and it suffices to show that if two relations are
  mutually included, then they are univalent.
  
  \subsubsection{Mere Propositions}
  
  As we mentioned earlier, proof irrelevance is a desired property in most systems. In HoTT,
  one distinguish between mere propositions and other types, where mere propositions
  are defined by:
  
  \Agda{ProofIrrel}{isprop-def}
  
  This allows us to state some very useful properties, which allows us to handle propositions
  as both true or false, depending on whether or not they're inhabited. These corresponds to
  lemma 3.3.2 in \cite{hottbook}.
  
  \Agda{ProofIrrel}{lemma-332}\\
  \Agda{ProofIrrel}{not-lemma-332}
  
  Which are both provable in Agda, but the proofs are omitted here.
  
  \subsubsection{Adding Relations to the mix}
  
  Now, to exploit such finer treatment of equality in our favor, we need to add
  relation and a few other details to the mix. We'll keep the relation definition
  as before, and push to the user the responsibility of proving that his relations
  are both mere propositions and decidable.
  
  This can be easily done with Agda's instance mechanism:
  
  \Agda{ProofIrrel}{instances}
  
  This will treat both records as typeclasses in the Haskell sense. Now, for talking about
  subrelations they must be an instance of \D{IsProp}, and whenever we wish to use anti-symmetry
  they must also be an instance of \D{IsDec}, and it turns out that anti-symmetry is now
  provable as long as we assume relational extensionality.
  
  \Agda{ProofIrrel}{subrel-def}\\
  \Agda{ProofIrrel}{subrel-antisym}
  
  Although we could arrive at the result we desired with a minimal number of postulates (univalence axiom, only), 
  the user was heavily punished for when he wants to define a relation, not to
  say that decidability will give problems once relational composition enters the stage. For this reasons
  we chose not to adopt this solution \emph{as is}, even though it's more formal than what we previously
  presented.

\subsubsection{Custom Universes}


  We could remove the \D{IsProp} record from our code, if we gave relations a bit more structure,
  and, prove that every object in this new (more structured) world is a mere proposition.
  One good option would be to encode a universe $U$ of mere propositions and have relations
  defined as $B \rightarrow A \rightarrow U$. This extra structure allows us to prove proof irrelevance
  for all $u : U$ (which holds by construction, once $u$ is a mere proposition), but only let's us define
  relations over $U$, which is less expressive than $Set$. The additional code boilerplate is also big,
  once we have to define a language and all operations that we'll need to perform with it.
  
  A universe and it's interpretation are defined as:
  
  \Agda{Universes}{data-U}
  
  Where $\mid\mid\_\mid\mid$ is a propositional truncation. That is, for every type $A$,
  there is a type $\mid\mid A \mid\mid$ with two constructors: (i) for any $x : A$ we have
  $\mid x \mid : \mid\mid A \mid\mid$; (ii) for any $x , y : \mid\mid A \mid\mid$, we have
  $x \equiv y$. This is also called smashing sometimes. Not every type constructor 
  preserves mere propositions. A simple example is with the coproduct
  $1 + 1$. Even though $1$ is a mere proposition, $1 + 1$ is not, since the elements of such type
  contain also information about which injection was used; We need to smash this information out
  if we want to keep with mere propositions.
  
  It turns out that we're just defining the logic we'll need to define relations, but this
  structure is very helpful! Now we can prove that given $u : U$, $\sharp u$ is a mere proposition.
  
  \Agda{Universes}{uprop-decl}
  
  So far so good! We just removed one instance from the user and proving decidability can be very
  straight forward (but in a few, rare, complicated cases)! Well, the changes were not only for the best.
  A new, minor, problem is the verbosity introduced by $U$, everything is harder to write and read.
  But there's a bigger situation happening here. If we look at the existential quantifier defined in $U$,
  it's witnesses must also come from $U$. This can be very restrictive once we start using relational
  composition (which is defined in terms of an existential).

\subsubsection{The Equality Model}
  
  Given the options discussed above, with all their positive and negative aspects, it seems
  a little too restrictive to adopt only one option. We indeed mixed both the $\releq$ promote
  with $\subseteq-antisym$. The idea is that the user not only can chose how formal he wants his model
  to be, but this can significantly increase development speed. In the first stages of development, where
  the base relations might change (and, if instances were written, they would consequently change) a lot,
  the user can keep developing with the $\releq$ promotion. Once his model is stable, he can completely
  formalize it by adding the desired instances and using subrelation anti-symmetry.

\section{Constructions}

\begin{TODO}
  \item relational fold!
\end{TODO}

After establishing a model for relations and relational equality, we follow by presenting
some of the important constructions here. Note that contrary to \emph{pen and paper} Mathematics,
we provide an encoding of the constructions and then we prove that our encoding satisfy the
universal property for the given construction, instead of adopting such property as the
definition. This not only proves the encoding to be correct, but it's the only constructive approach
we can use.

The design adopted for the lower level constructions is somewhat heavy in it's notation. The
main reason for this choice (which differs significantly from other Relational Algebra implementations)
is it's ease of use when coupled with reflection techniques. If we provide all definitions
as Agda functions, when we access a term AST, Agda will normalize and expand such definitions.
By encapsulating it in records, we can stop this normalization process and use a (much) smaller
AST representation.

\subsection{Products}

Given two relations $R : A \rightarrow B$ and $S : A \rightarrow C$, we can construct a relation
$\langle R , S \rangle : A \rightarrow B \times C$ such that $(b,c)\;\langle R , S \rangle\;a$ if and only if $b\;R\;a \wedge c\;S\;a$.
Without getting in too much detail of what it means to be a product, we usually write
it in the form of a commutative diagram:

\begin{displaymath}
\xymatrix{%
B & B \times C \ar[l]_{\pi_1} \ar[r]^{\pi_2} & C \\
  &     A \ar@{..>}[u]|{\langle R , S \rangle} \ar[ul]^{R} \ar[ur]_{S}
  &
}
\end{displaymath} 

That is,
\begin{eqnarray*}
  R &=& \pi_1 \cdot \langle R , S \rangle \\
  S &=& \pi_2 \cdot \langle R , S \rangle
\end{eqnarray*}

Products are unique up to isomorphism, which explains the notation without introducing any new names.
The proof is fairly simple and can be conducted by \emph{gluing} two product diagrams.
The diagrammatic notation states the existence of a relation $\langle R , S \rangle$ and the dotted arrow
states it's uniqueness. $\pi_1\;(b , c) = b$ and $\pi_2\;(b , c) = c$ are the canonical
projections.\\

\begin{mydef}[Split Universal]
Given $R$ and $S$ as above, let $X : A \rightarrow B \times C$, then:
\[
  X \subseteq \langle R , S \rangle \Leftrightarrow \pi_1 \cdot X \subseteq R \wedge \pi_2 \cdot X \subseteq S
\]\\
\end{mydef}
Encoding this in Agda is fairly simple, once we already have products (in their categorical sense)
available. We just wrap everything inside a record:

\Agda{RelationsCore}{product-final}

It's universal property can be derived from the following \emph{lemmas}

\Agda{RelationsCore}{product-univ-r1}\\
\Agda{RelationsCore}{product-univ-r2}\\
\Agda{RelationsCore}{product-univ-l}

In fact, the product of relations respects both decidability and propositionality (that is,
given two mere propositions, their product is still a mere proposition). Therefore, such 
instances are already defined.

\subsection{Coproduct}

If we flip every arrow in the diagram for products, we arrive at it's dual notion, usually
called coproduct or sum. Given two relations $R : B \rightarrow A$ and $S : C \rightarrow A$,
we can perform a \emph{case analysis} in an element of type $B + C$ and relate it to an $A$.
Indeed, the \emph{either} of $R$ and $S$, denoted $[R , S]$, has type $B + C \rightarrow A$
and is depicted in the following diagram:

\begin{displaymath}
\xymatrix{%
 B \ar[r]^{\iota_1} \ar[dr]_{R} & B + C \ar@{..>}[d]|{[R , S]} & C \ar[l]_{\iota_2} \ar[dl]^{S} \\
   &   A   &
}
\end{displaymath}

\begin{TODO}
  \item Put the agda definitions here.
  \item Present the problem with propositionality
\end{TODO} 


\subsection{Composition}

Given two relations $R : B \rightarrow C$ and $S : A \rightarrow B$, we can construct a
relation $R \cdot S : A \rightarrow C$, read as $R$ after $S$, and defined by:
\[
  R \cdot S = \{ (a, c) \in A \times C\;\mid\; \exists b \in B\;.\; a\;S\;b \wedge b\;R\;c \}
\]
Or, using diagrams:
\begin{displaymath}
  \xymatrix{ A \ar[r]^S \ar@/_1pc/[rr]_{R \cdot S} & B \ar[r]^R & C }
\end{displaymath}

As a first definition in Agda, one would expect something like:

\Agda{RelationsCore}{composition-naive}

And here we can start to see the dependent types shining. The existential quantification
is just some syntax sugar for a dependent product. Therefore, for constructing a composition
we need to provide a witness of type $B$ and a proof that $c R b \wedge b S a$, given $c$ and $a$.

Yet, this suffers the problem we mentioned earlier. Agda will normalize every occurrence of \F{\_after\_}
to it's right-hand side, which is a non-linear lambda abstraction, and will make subterm matching 
very complex to handle. An option is to use the exact definition of an existential quantifier,
but expand it:

\Agda{RelationsCore}{composition-final}








  
%\chapter{Notes on Reflection}
%\label{chap:reflection}
%
Agda introduced a reflection API in version 2.2.8. Although not a new feature
in the world of functional programming (Lisp's \emph{quoting} and \emph{unquoting} and Template Haskell, for instance, are similar techniques) 
it is proving to be very usefull for implementing interesting things. One application for reflection, which I chose to explore, is the possibility to write
custom tactics for proving propositions. Very similar to what Coq does.

In Agda, the representation of a term is something with type \AgdaType{Term}.
The keywords that bridge the world of programs and their representations are\AgdaKeyword{quoteTerm} and \AgdaKeyword{unquote}. They can be seen as inverses of each other. The former transforms a
normal term into it's \AgdaType{Term} representation whereas the later does exactly the oposite. I'll give a brief introduction to the reflection API and
some simple examples of how could we manipulate terms. I'll also discuss some
of the difficulties one might encounter while working with reflection.

\section{A Simple Example}

aa


\chapter{Summary and Future Work}
\label{chap:futurework}
The overall task of adding rewriting functionality to Agda is quite an exploratory 
project for a couple reasons, the most important being the unstable state of Agda's
standard library and, in particular, of the Reflection module. 

The work started by getting familiar both with the Agda language and the required theoretical background. 
Our case study is the development of a library for Relational Algebra, 
that will be used as our main target for the rewriting functionality.
During this development I stumbled across a few problems regarding the notion of equality, 
which were solved using a combination of techniques, as explained in section \ref{chap:relationalalgebra}.
This significantly slowed down the project, mostly because one had to be sure to get
this right, otherwise the library wouldn't be suitable for automatic rewriting, as is the case
of current state-of-the-art Relational Algebra libraries, in Agda.

The library is far from complete, as any library requires constant
management and adaptation, but is stable enough for the work to proceed to the next task, which is
exploring the reflection capabilities of Agda, and exploring how to use them to provide 
a general rewriting functionality. This is the form of the current work.

Once a reliable rewriting mechanism is ensured, we would like to run some bigger verification
tasks, in order to \emph{destructively} test our tool. Parallel to this, there is also the
work of expanding the library with more Relational Algebra constructions. As mentioned before,
this is constant work, at least until Agda's stdlib stabilizes. We're currently working with
Agda version $2.4.2.2$ with standard library version $0.9$. The code we developed so far and
the sources for this document are available at the git repository \texttt{https://github.com/VictorCMiraldo/msc-agda-tactics}.

The use of this library to support the relational derivation of functional programs in
the algebra of programming style is the ultimate goal of this project, once the technical
difficulties mentioned above are addressed and solved.


\bibliographystyle{alpha}
\bibliography{references}

\end{document}
